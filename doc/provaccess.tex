

\subsection{provenance DataModel serialization}
There are three possible families of ProvDM metadata serialization
\begin{itemize}
 \item W3C serializations : Prov-N, PROV\-Json, PROV\-XML. These formats allow the possibility to add additional IVOA or ad hoc attributes to the basic ones in each class.
 \item Mapping of ProvDM classes onto tables with appropriate relationships. This can allow managment by a TAP service (the model mapping is then described with the TAP schema). The serialization will be a single table according to the query.

 \TODO{TAP SCHEMA of the ProvDM datamodel: Maybe Matthieu can provide us with a copy of the TAP schema he designed ?}
  \item Direct VOTABLE mapping by using   some ad hoc mapping based on transcription of PROV-N format : this is called PROV-VOTABLE. Moreover in the future we could also define a VO-DML mapping (ref) version of the mapping.
The following is an example of provenance metadata in this PROV-VOTABle format. Objects become Tables the class of which is rendred by a utype. Attributes and relationships become FIELDS or PARAMS. The model attribute names also become VOTABLE utypes.  
\begin{verbatim}
<TABLE name="cta:telescope_stage_520" utype="prov:activity" >
     <PARAM name="start" utype="prov:startTime" datatype="char" arraysize="*" xtype="ISO8601" value="2015-07-30T09:45:00" >
     <PARAM name ="stop" utype="prov:endTime"  datatype="char" arraysize="*" xtype="ISO8601" value = "2015-07-30T10:00:00" >
     <PARAM name="methodname" utype="voprov:method_name" dataype="char" arraysize="*" value="Telescope_stage" >
     <PARAM name="version" utype="voprov:method_version" datatype="char" arraysize="*" value="1.0" >    
     <PARAM utype="voprov:used" datatype="char" arraysize="*" value="cta:run13000_EVT0" >
     <PARAM utype="voprov:used" datatype="char" arraysize="*" value="cta:Stage1Config_5250" >    
</TABLE>
<TABLE name="cta:Stage1Config_5250", utype="prov:entity" >
    <PARAM name="type" utype="prov:type" datatype="char" arraysize="*" value="file" >   
</TABLE>
<TABLE name="cta:run1000_EVT1", utype="prov:entity" >
      <PARAM name="label" utype="prov:label"datatype="char" arraysize="*" value="EVT1 file" >
      <PARAM name="type" utype="prov:type" datatype="char" arraysize="*" value="file" >
      <PARAM name="run" utype="cta:runNumber" datatype="int"  value="13000" >
      <PARAM name="tel" utype="cta:telescope" datatype="char" arraysize="*" value="MST21" >
      <PARAM utype="wasGeneratedBy"  datatype="char" arraysize="*" value="cta:Stage1Config_5250">
</TABLE>
\end{verbatim}
  
  
\end{itemize}
\subsection{access protocols}
\begin{itemize}
\item ProvDAL: retrieve provenance information based on given id of a dataEntity or activity

ProvDAL is a service the interface of which is organized around one main PARAMETER, the "ID" of the entity (obs\_publisher\_did of an ObSDataSet for example) The response is given in one of the following formats: PROV-N, PROV-JSON, PROV-XML, PROV-VOTABLE. Additional parameters can complete ID to refine the query. FORMAT allows to choose the output format. STEP allows to discriminate between STEP=LAST which gives the last step in the provenace chain and STEP = ALL which gives the whole chain.
Multiple ID PARAMETER is allowed in order to retrieve several data set provenance details at the same time.
\item ProvTAP: allows detailed queries for provenance information, discovery of datasets based on 
e.g. code version.

ProvTAP is a TAP service implementing the ProvDM datamodel. The PROVDM  mapping is included in the TAP schema (see above). The result of any query is a single table joigning information coming from one or several "provenace" tables available in the database. 

A special case is considered where ProvDM and OBscore are both implemented in the same TAP service and queried together. The TAP response is then providing an Obscore Table with a ProvDM extension. We can imagine that in the future this could be hard-coded and registered as an ObsTAPRov service. 


\item Do we need combined query possibilities, i.e. ask for ObsCore-fields and Provenance fields
in one query? Or rather use a 2-step-process, decoupling them from each other?
\end{itemize}


\TODO{Also look at PROV-AQ from the W3C.}
