
%\newcolumntype{Y}{>{\raggedright\arraybackslash}X}

%\TODO{ML: here I suggest to split the serialisation description from the Access part in two different sections and not subsections  }

\section{Provenance Data Model serialization}
\label{sec:serialisations}
\subsection{Serializing of the datamodel core}
\label{sec:intro-serialization}
The serialization files documents constitutes the building blocks of the client/server dialogs.

The provenance information as represented in the data model is split in three main concepts that can be searched following many different relations involved between the main 3 classes. 
The selection of the relations to expose when distributing the provenance information depends on the usage and will be described more extensively in the Use-case sections (\ref{sec:usecases-implementations}) , the implementation note \citep[]{std:ProvenanceImplementationNote} and the links therein.

To give a very simple example, suppose a client asks for the context of execution for one specified Activity, which computes a simple RGB color composition. 

On the server side, exposing the Provenance information for this Activity or for an Entity, corresponding to a monocolor or RGB image, is just exposing the structure of the classes and relation tables and feed them with the related t-upples in the database.
On the client side, the content of a VO-Provenance serialization document can then be explored and represented using graphical interfaces, as inspired by the Provenance Southampton suite or by customized visualization tools.
 
\subsection{W3C Serialisation Formats reused and exented}
In the W3C Provenance framework, three descriptions formats are proposed to serialize the Provenance metadata : {PROV-N}, {PROV-JSON}, {PROV-XML} as defined in \citep[]{std:W3CProvN}, \citep[]{soton356855}. These are serializations of the W3C provenance data model, a larger set of classes and relations compared to this model but sharing the same core structure. They allow the possibility to add IVOA or \textit{ad hoc} attributes to the basic ones in each class. This way the IVOA models can produce W3C compliant serializations and take benefit of W3C visualizing tools.

%example KR provn 
Here is a serialization instance document for an entity being processed by an activity, in PROV-N format:

\begin{verbnobox}[\scriptsize]

document
  prefix ivo <http://www.ivoa.net/documents/rer/ivo/>
  prefix ex <http://www.example.com/provenance/>
  prefix voprov <http://www.ivoa.net/documents/dm/provdm/voprov/>

  entity(ivo://example#Public_NGC6946, [voprov:name="Processed image of NGC 6946"])
  entity(ivo://example#DSS2.143, [voprov:name="Unprocessed image of NGC 6946"])
  activity(ex:Process1, 2017-04-18T17:28:00, 2017-04-19T17:29:00, [voprov:name="Process 1"])
  used(ex:Process1, ivo://example#DSS2.143, -)
  wasGeneratedBy(ivo://example#Public_NGC6946, ex:Process1, 2017-05-05T00:00:00)
endDocument

\end{verbnobox}

%example KR provjson (entity, agent and activity)

\begin{verbnobox}[\scriptsize]
{
  "prefix": {
    "ivo": "http://www.ivoa.net/documents/rer/ivo/",
    "voprov": "http://www.ivoa.net/documents/dm/provdm/voprov/",
    "ex": "http://www.example.com/provenance/"
  },
  "activity": {
    "ex:Process1": {
      "prov:startTime": "2017-04-18T17:28:00",
      "prov:endTime": "2017-04-19T17:29:00",
      "voprov:name": "Process 1"
    }
  },
  "wasGeneratedBy": {
    "_:id4": {
      "prov:time": "2017-05-05T00:00:00",
      "prov:entity": "ivo://example#Public_NGC6946",
      "prov:activity": "ex:Process1"
    }
  },
  "used": {
    "_:id1": {
      "prov:entity": "ivo://CDS/P/DSS2/POSSII#POSSII.J-DSS2.143",
      "prov:activity": "hips:AlaRGB1"
    }
  }
  "entity": {
    "ivo://example#DSS2.143": {
      "voprov:name": "Unprocessed image of NGC6946"
    },
    "ivo://example#Public_NGC6946": {
      "voprov:name": "Processed image of NGC 6946"
    }
  }
}
\end{verbnobox}
\subsection{Prov-VOTable format} 
To emphasize the compatibility to the IVOA framework, where the VOTable-XML format is a reference to circulate metadata, we define a PROV-VOTABLE mapping specification. All classes' declarations and relations described in PROV-N are translated as separated tables, one for each class of the model.
All attributes of these classes are translated as columns, i.e, VOTable FIELDS. 
In addition, the specification defines the VOTable values of FIELD and PARAM attributes ucd, datatype, utype, unit, description, etc. 

This can be appropriately used for two goals:
\begin{itemize}
	\item publishing full provenance metadata for data collections in VOTable format. This can be produced by data processing workflows or as output of databases containing provenance metadata.
	\item providing the backbone for the TAP Schema describing IVOA provenance metadata which we call ProvTAP 
\end{itemize}

These VOTable serialisations can be produced  using  the VOPROV Python module \footnote{\url{https://github.com/sanguillon/voprov}} python module, available to the community. See also Section~\ref{sec:implementation_voprov} and the IVOA Prov-DM Implementation Note \citep[]{std:ProvenanceImplementationNote}. 

%example KR prov VOTABLE .
Here is VOTable document transcription of the corresponding serialization example given above in prov-N and Prov-Json.

\begin{verbnobox}[\scriptsize]
<?xml version="1.0" encoding="UTF-8"?>
<VOTABLE version="1.2" xmlns="http://www.ivoa.net/xml/VOTable/v1.2" xmlns:ex="http://www.example.com/provenance" xmlns:ivo="http://www.ivoa.net/documents/rer/ivo/" xmlns:voprov="http://www.ivoa.net/documents/dm/provdm/voprov/" xmlns:xsi="http://www.w3.org/2001/XMLSchema-instance" xsi:schemaLocation="http://www.ivoa.net/xml/VOTable/v1.2 http://www.ivoa.net/xml/VOTable/VOTable-1.2.xsd">
  <RESOURCE type="provenance">
    <DESCRIPTION>Provenance VOTable</DESCRIPTION>
    <TABLE name="Usage" utype="voprov:used">
      <FIELD arraysize="*" datatype="char" name="activity" ucd="meta.id" utype="voprov:Usage.activity"/>
      <FIELD arraysize="*" datatype="char" name="entity" ucd="meta.id" utype="voprov:Usage.entity"/>
      <DATA>
        <TABLEDATA>
          <TR>
            <TD>ex:Process1</TD>
            <TD>ivo://example#DSS2.143</TD>
          </TR>
        </TABLEDATA>
      </DATA>
    </TABLE>
    <TABLE name="Generation" utype="voprov:wasGeneratedBy">
      <FIELD arraysize="*" datatype="char" name="entity" ucd="meta.id" utype="voprov:Generation.entity"/>
      <FIELD arraysize="*" datatype="char" name="activity" ucd="meta.id" utype="voprov:Generation.activity"/>
      <DATA>
        <TABLEDATA>
          <TR>
            <TD>ivo://example#Public_NGC6946</TD>
            <TD>ex:Process1</TD>
          </TR>
        </TABLEDATA>
      </DATA>
    </TABLE>
    <TABLE name="Activity" utype="voprov:Activity">
      <FIELD arraysize="*" datatype="char" name="id" ucd="meta.id" utype="voprov:Activity.id"/>
      <FIELD arraysize="*" datatype="char" name="name" ucd="meta.title" utype="voprov:Activity.name"/>
      <FIELD arraysize="*" datatype="char" name="start" ucd="" utype="voprov:Activity.startTime"/>
      <FIELD arraysize="*" datatype="char" name="stop" ucd="" utype="voprov:Activity.endTime"/>
      <DATA>
        <TABLEDATA>
          <TR>
            <TD>ex:Process1</TD>
            <TD>Process 1</TD>
            <TD>2017-04-18 17:28:00</TD>
            <TD>2017-04-19 17:29:00</TD>
          </TR>
        </TABLEDATA>
      </DATA>
    </TABLE>
    <TABLE name="Entity" utype="voprov:Entity">
      <FIELD arraysize="*" datatype="char" name="id" ucd="meta.id" utype="voprov:Entity.id"/>
      <FIELD arraysize="*" datatype="char" name="name" ucd="meta.title" utype="voprov:Entity.name"/>
      <DATA>
        <TABLEDATA>
          <TR>
            <TD>ivo://example#DSS2.143</TD>
            <TD>Unprocessed image of NGC6946</TD>
          </TR>
          <TR>
            <TD>ivo://example#Public_NGC6946</TD>
            <TD>Processed image of NGC 6946</TD>
          </TR>
        </TABLEDATA>
      </DATA>
    </TABLE>
    <INFO name="QUERY_STATUS" value="OK"/>
  </RESOURCE>
</VOTABLE>

\end{verbnobox}

This VOTable serialization can be considered as a flat view on the various tables stored in a database implementing the datamodel structure explained in Section~\ref{sec:datamodel}.
More examples of serialization documents are provided in Appendix \ref{sec:appendix-serialization-examples}.
 

Such serializations can be retrieved through access protocols (see \ref{sec:access_protocols} ) or directly integrated in dataset headers or ``associated metadata'' in order to provide provenance metadata for these datasets. E.g. for FITS files a provenance extension called ``PROVENANCE'' could be added which contains provenance information of the workflow that generated the FITS file in one of the serialization formats.
\TODO{Check that this keyword is not already taken.}
\TODO{SVOM strategy to incorporate provenance as an extension in FITS ? still valid ?}


\subsection{Serialization of description classes}
\label{sec:description-serialization}
%{updated by Mathieu} 
The ProvenanceDM includes description classes that can exist before any provenance information is recorded. First, the ActivityDescription class gives information on the activity (name, description, doculink...) and the parameters expected as an input. In addition, UsedDescription and WasGeneratedByDescription classes indicate the expected roles of the input and output entities respectively. Finally, The activity may expect specific kinds of entities as inputs or outputs, for which there may be detailed descriptions stored as EntityDescription records.

The serialization of an ActivityDescription, that includes all those description classes, is based on the IVOA DataLink Service Descriptors for service resources \citep{std:Datalink}, and can thus be stored as a VOTable  \citep{std:VOTABLE}. Indeed, a service descriptor points to a service that probably executes an activity using the given input parameters, some of which probably point to entities. One can thus easily translate an ActivityDescription VOTable to a DataLink service descriptor VOTable block, and vice-versa. 

The VOTable contains one resource with attributes type=``meta'' and utype=``voprov:ActivityDescription''. This resource contains PARAM elements to describe the activity and GROUP elements with additional PARAM elements to describe the input parameters (group name=``InputParams''), the input entities (group name=``Used'') and the output entities (group name=``Generated''). 

The standard PARAM elements for an activity resource correspond to the attributes of the ActivityDescription class (see Section~\ref{sec:activity}) and may include an Agent name and email. For the input parameters, each ParameterDescription element is mapped to a PARAM element. The mapping is direct as ParameterDescription is based on PARAM. For the input and output entity groups, each related entity is described with a PARAM block where the name is the role of the entity in the scope of the activity, and the expected value is the entity identifier (utype=``voprov:Entity.id''). It is possible to reference an input parameter using the ref attribute of PARAM, if an input entity is given as an input parameter to the activity (e.g. the name of a file). The xtype attribute of PARAM can be used to provide the content type (MIME type) of the entity.

Here is an example of an ActivityDescription VOTable that describes an activity to create an RGB image from three red, green, blue images:


\begin{verbnobox}[\scriptsize]

<VOTABLE xmlns:xsi="http://www.w3.org/2001/XMLSchema-instance" 
    xmlns="http://www.ivoa.net/xml/VOTable/v1.3" version="1.3" 
    xsi:schemaLocation="http://www.ivoa.net/xml/VOTable/v1.3 
    http://www.ivoa.net/xml/VOTable/v1.3">
  <RESOURCE ID="make_RGB_image" name="make_RGB_image" 
      type="meta" utype="voprov:ActivityDescription">
    <DESCRIPTION>Create an RGB image from 3 images</DESCRIPTION>
    <LINK content-role="doc" href="..."/>
    <PARAM name="label" datatype="char" arraysize="*" 
        value="make_RGB_image" utype="voprov:ActivityDescription.label"/>
    <PARAM name="type" datatype="char" arraysize="*" 
        value="None" utype="voprov:ActivityDescription.type"/>
    <PARAM name="subtype" datatype="char" arraysize="*" 
        value="None" utype="voprov:ActivityDescription.subtype"/>
    <PARAM name="version" datatype="float" 
        value="None" utype="voprov:ActivityDescription.version"/>
    <PARAM name="contact_name" datatype="char" arraysize="*" 
        value="..." utype="voprov:Agent.name"/>
    <PARAM name="contact_email" datatype="char" arraysize="*" 
        value="...@..." utype="voprov:Agent.email"/>
    <GROUP name="InputParams" utype="voprov:Parameter">
      <PARAM ID="RGB" arraysize="*" datatype="char" name="RGB" 
          type="no_query" value="RGB.jpg">
        <DESCRIPTION>RGB image name</DESCRIPTION>
      </PARAM>
      <PARAM ID="order" arraysize="*" datatype="char" name="order" 
          type="no_query" value="RGB">
        <DESCRIPTION>order of the channels</DESCRIPTION>
        <VALUES>
          <OPTION value="RGB"/>
          <OPTION value="RBG"/>
          <OPTION value="GBR"/>
          <OPTION value="GRB"/>
          <OPTION value="BRG"/>
          <OPTION value="BGR"/>
        </VALUES>
      </PARAM>
    </GROUP>
    <GROUP name="Used" utype="voprov:Used">
      <PARAM arraysize="*" datatype="char" name="R" 
          value="R.jpg" utype="voprov:Entity.id" xtype="image/jpeg">
        <DESCRIPTION>Image for red channel</DESCRIPTION>
      </PARAM>
      <PARAM arraysize="*" datatype="char" name="G" 
          value="G.jpg" utype="voprov:Entity.id" xtype="image/jpeg">
        <DESCRIPTION>Image for green channel</DESCRIPTION>
      </PARAM>
      <PARAM arraysize="*" datatype="char" name="B"
          value="B.jpg" utype="voprov:Entity.id" xtype="image/jpeg">
        <DESCRIPTION>Image for blue channel</DESCRIPTION>
      </PARAM>
    </GROUP>
    <GROUP name="Generated" utype="voprov:WasGeneratedBy">
      <PARAM arraysize="*" datatype="char" name="RGB" ref="RGB"
          value="RGB.jpg" utype="voprov:Entity.id"  xtype="image/jpeg">
        <DESCRIPTION>RGB image name</DESCRIPTION>
      </PARAM>
    </GROUP>
  </RESOURCE>
</VOTABLE>

\end{verbnobox}

\section{Accessing provenance information}
\subsection{Access protocols}
\label{sec:access_protocols}
We envision two possible access protocols:
\begin{itemize}
\item ProvDAL: retrieve provenance information based on given ID of a data entity or activity.
\item ProvTAP: allows detailed queries for provenance information, discovery of datasets based on e.g. code version.
\end{itemize}

\subsection{ProvDAL}
\subsubsection{ProvDAL}
ProvDAL is a service the interface of which is organized around one main parameter, the \urlparam{\bf ID} of an entity (obs\_publisher\_did of an ObsDataSet for example), activity or an agent.
The response is given in one of the following formats: \urlparam{PROV-N}, \urlparam{PROV-JSON}, \urlparam{PROV-XML}, \urlparam{PROV-VOTABLE}.
Additional parameters can complete the \urlparam{ID} to refine the query: \urlparam{\bf FORMAT} allows to choose the output format. \urlparam{\bf DEPTH} gives the number of relations that shall be tracked along the provenance history, independent of the type of relation. Its value is either 0, a positive integer or \urlparam{ALL}. If this parameter is omitted, the default is \urlparam{ALL}, which returns the complete provenance history that the service has stored or the provenance according to a maximum depth number that the server allows.

The \urlparam{ID} parameter is allowed more than once in order to retrieve provenance details for several activities or datasets at the same time. Here are a few example requests:

\begin{verbatim}
{provdal-base-url}?ID=rave:dr4&FORMAT=PROV-JSON
{provdal-base-url}?ID=rave:dr4&ID=rave:act_irafReduction&DEPTH=2
\end{verbatim}

\noindent
The format can also be specified via the HTTP accept header, e.g.
\begin{verbatim}
wget -d --header="Accept: application/json" \
   {provdal-base-url}?ID=rave:dr4
\end{verbatim}
would return the provenance information in \urlparam{PROV-JSON} format.
\noindent
If both \urlparam{FORMAT} and the accept header are used and \urlparam{FORMAT} specifies a format that is incompatible with the HTTP accept header, then the service should return with a HTTP status 406: Not Acceptable.

For services which allow tracking the provenance information forward, e.g. in order to check for which activities an entity was used, the optional parameter \urlparam{\bf DIRECTION} can be set to \urlparam{FORTH}. Its default value is \urlparam{BACK}. This influences the direction in which the used, wasGeneratedBy, wasDerivedFrom and wasInfluencedBy relations are followed.

The provenance data model defines also the hierarchical relations \emph{hadMember} for entity collections and \emph{hadStep} for activityFlows. If a node belongs to a collection or activityFlow, these relations shall be returned as well, independent of the specified tracking direction.
If one is interested in more details and wants to follow the \emph{members} of an entity collection or the \emph{steps} of an activityFlow, these can be included by setting the optional parameter \urlparam{\bf MEMBERS} or \urlparam{\bf STEPS} to \urlparam{TRUE}, respectively. The default is \urlparam{FALSE}.

By default, it is recommended to stop any further tracking at an agent node, unless an additional optional parameter \urlparam{\bf AGENT} is set to \urlparam{TRUE}. Note that this means that the request for any agent will always return just the agent node itself and nothing else, unless \urlparam{AGENT=TRUE} is used. Thus, if one wants to know which entities and activities an agent has influenced, the request looks like this:

\begin{verbatim}
{provdal-base-url}?ID=org:rave&AGENT=TRUE&DEPTH=1
\end{verbatim}

\noindent
\urlparam{DEPTH=1} was used here in order to avoid following the found entities and activities any further.

%\comment{Maybe it's better to use DEPTH and DIRECTION instead of FORWARD and BACKWARD. Reason: if a service just implements the backward direction, then it's weird to call something ``backward'' if there is no ``forward'' as well. DEPTH is also a commonly used word when refering to graphs and numbers of relations.}


\begin{figure}[h]
\centering
\includegraphics[width=1.0\textwidth]{provenance-graph-example-depth2.pdf}
\caption{An example provenance graph, highlighting the objects and relations returned from a ProvDAL service with ID=E6 and \urlparam{DEPTH}=2. The \urlparam{BACK} and \urlparam{FORTH} values for \urlparam{DIRECTION} are only important for the processing relations (solid lines). Hierarchial (dashed) and responsibility relations (dotted) are only followed ``upwards'' and towards agents by default. If they should also be followed in the other direction, then the additional optional parameters \urlparam{MEMBERS}, \urlparam{STEPS} and \urlparam{AGENT} need to be set to \urlparam{TRUE}.}
\label{fig:provenance-graph-example}
\end{figure}


A ProvDAL service MUST implement the parameters \urlparam{ID}, \urlparam{DEPTH} and \urlparam{FORMAT}; the remaining parameters are optional.
If a service does not implement the optional parameters, but they appear in the request, then the service should return with an error.

Table~\ref{tab:provdal-parameters} summarizes the parameters for such a ProvDAL service interface.

\begin{table}[h]
\small
\begin{tabulary}{1.0\textwidth}{@{}p{0.17\textwidth}p{0.22\textwidth}p{0.53\textwidth}@{}}
%{llp{0.2\textwidth}p{0.3\textwidth}}
\toprule
\head{Parameter} & \head{Value/options} & \head{Description}\\\hline
\midrule
\textbf{\urlparam{ID}} & qualified \urlparam{ID} & a valid qualified identifier for an entity or activity (can occur multiple times)\\
\textbf{\urlparam{DEPTH}} & 0,1,2,..., \urlparam{\underline{ALL}} &  number of relations to be followed or \texttt{ALL} for everything, independent of the relation type\\
\textbf{\urlparam{FORMAT}} & \urlparam{PROV-N}, \newline\urlparam{PROV-JSON}, \newline\urlparam{PROV-XML}, \newline\urlparam{PROV-VOTABLE} & serialisation format of the response\\\hline
\urlparam{DIRECTION} & \urlparam{\underline{BACK}}, \urlparam{FORTH} & \urlparam{BACK} = track the provenance history, \newline\urlparam{FORTH} = explore the results of activities and where entities have been used\\
\urlparam{MEMBERS} & \urlparam{TRUE} or \urlparam{\underline{FALSE}} & if \urlparam{TRUE}, retrieve and track members of collections\\
\urlparam{STEPS} & \urlparam{TRUE} or \urlparam{\underline{FALSE}} & if \urlparam{TRUE}, retrieve and track steps of activityFlows\\
\urlparam{AGENT} & \urlparam{TRUE} or \urlparam{\underline{FALSE}} & if \urlparam{TRUE}, retrieve all relations for agents, i.e. find out what an agent is responsible for\\
\bottomrule
\end{tabulary}
\caption{ProvDAL request parameters. Options that are \textbf{required} to be implemented by ProvDAL services are marked with bold face. \underline{Default} values are underlined.}
\label{tab:provdal-parameters}
\end{table}




\subsection{ProvTAP}   //currently updated ...
ProvTAP is a TAP service implementing the ProvenanceDM data model. The data model mapping is included in the TAP schema. The mapping of ProvenanceDM classes and attributes onto tables and columns of the schema with the appropriate relationships, datatypes, units, utypes and ucds is done similarly to the PROV-VOTABLE serialization. The query response will result in a single table according to the query.
This single table is joining information coming from one or several ``provenance'' tables available in the database.

A special case is considered where ProvenanceDM and ObsCore are both implemented in the same TAP service and queried together. The TAP response is then providing an Obscore table with a ProvenanceDM extension. We can imagine that in the future this could be hard-coded and registered as an ObsProvTAP service.


\TODO{We need more details here! Output of TAP service is NOT a PROV-VOTABLE by default!}

%\TODO{Do we need combined query possibilities, i.e. ask for ObsCore-fields and Provenance fields in one query? Or rather use a 2-step-process, decoupling them from each other?}


%\TODO{Also look at PROV-AQ from the W3C.}


\subsection{VOSI availability and capabilities}
\TODO{Still needs to be discussed!}
According to the DALI specification for VO services \citep{std:DALI}, a provenance service implementing ProvDAL and/or ProvTAP must provide a VOSI availability interface as well as a capabilities interface with entries for ProvDAL and/or ProvTAP. The \texttt{standardId}s for these provenance interfaces are:

\begin{verbatim}
ivo://ivoa.net/std/ProvenanceDM#ProvDAL
ivo://ivoa.net/std/ProvenanceDM#ProvTAP
\end{verbatim}

The capability for a TAP service to support the Provenance DM is expressed by the 
dataModel element as :
\begin{verbatim}
<dataModel ivoid="ivo://ivoa.net/std/ProvenanceDM#core-1.0">ProvenanceDM-1.0</dataModel>
\end{verbatim}

 For ProvTAP, the VOSI tables interface 
also needs to be provided.

