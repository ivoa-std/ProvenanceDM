In this document, we discuss a draft for an IVOA standard data model for
describing the provenance of data. We focus on observational data, since
provenance for simulated data is already covered by the Simulation Data Model
(SimDM \citep{std:SimDM}). However, the version currently discussed is
sufficiently abstract, so that its core pattern could be applied to any kind
of process, including extraction of data from 
databases or even the flow of scientific proposals from application to 
acceptance, including scheduling of the observations proposed therein.
Provenance information could also be used to check internal processes,
e.g., whether a proposal was approved by a person from a certain committee,
or whether the time span between application and acceptance or rejection
does not extend a certain period, etc.\,. 


\subsection{Goal of the provenance model}\label{sec:goals}
The goal of the provenance data model is to describe how provenance information
can be modeled, stored and exchanged within the Virtual Observatory. Its scope
is mainly modeling of the flow of data, of the relations between data,
and of processing steps. Characteristics of observations such as ambient
conditions and instrument characteristics will not be modeled here
explicitely. They can be included in form of additional data linked to
observations, or as attributes of observation processes.

In general, the model shall enable a scientist who has no prior knowledge about
a dataset to get more 
background information. This will help the scientist decide if the dataset 
is adequate for his research goal, judge its quality and get enough information
to be able to trace back its history as far as possible. 

Provenance information may be recorded in minute detail or by using coarser
elements, depending on the intended usage and the desired level of detail
for a specific project that records provenance.
