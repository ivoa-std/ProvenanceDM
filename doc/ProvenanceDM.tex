\documentclass[11pt,a4paper]{ivoa}
\input tthdefs

\usepackage[utf8]{inputenc}
\usepackage{booktabs, tabulary}    % for nicer tables

% make the text in pdf properly searchable
\usepackage{lmodern}

% use listings for including text files and code snippets
\usepackage{listings}

\title{IVOA Provenance Data Model}

\ivoagroup{DM}

\author{Kristin Riebe}
\author{Mathieu Servillat}
\author{François Bonnarel}
\author{Mireille Louys}
\author{Florian Rothmaier}
\author{Michèle Sanguillon}
\author{IVOA Data Model Working Group}

\editor{Kristin Riebe}
\editor{Mathieu Servillat}

% \previousversion[????URL????]{????Funny Label????}
\previousversion[http://www.ivoa.net/documents/ProvenanceDM/20161121/]{WD-ProvenanceDM-1.0-20161121.pdf}
\previousversion[http://volute.g-vo.org/svn/trunk/projects/dm/provenance/description/ProvDM-0.2-20160428.pdf]{ProvDM-0.2-20160428.pdf}
\previousversion[http://volute.g-vo.org/svn/trunk/projects/dm/provenance/description/ProvDM-0.1-20141008.pdf]{ProvDM-0.1-20141008.pdf}


% own definitions
\definecolor{todocolor}{rgb}{1,1,0.8}
\definecolor{darkred}{rgb}{0.6,0,0}
\definecolor{rose}{rgb}{1.0,0.88,0.88}
\definecolor{darkgrey}{rgb}{0.35,0.35,0.35}
%\newcommand{\TODO}[1]{%
%    \noindent%
%    \textcolor{todocolor}{\sffamily [\textbf{TODO:} #1]}%
%}

\newcommand{\TODO}[1]{%
    \noindent%
    \colorbox{todocolor}{%
            \parbox{0.85\linewidth}{\sffamily \textbf{TODO:}\\
            #1}
    }%
    \vspace{2pt}

}

\newcommand{\note}[1]{%
    \noindent%
    \textcolor{darkgrey}{{\sffamily Note:} \emph{#1}}%
}


\newcommand{\paragraphlb}[1]{\paragraph{#1}\mbox{}\\} % paragraph with line break

\setlength{\fboxsep}{5pt}
%\setlength{\fboxrule}{1.5pt}
\newcommand{\warning}[1]{%
    \vspace{\baselineskip}
    \noindent
    \parbox{\linewidth}{%
        \colorbox{darkred}{%
            \parbox{0.7\linewidth}{\large \sffamily \textcolor{white}{Warning}}%
        }\\[-1pt]
        \noindent%
        \fcolorbox{darkred}{rose}{%
            \parbox{0.7\linewidth-2\fboxrule}{#1}%
        }%
    }%
    \vspace{\baselineskip}
}%

% for nicer tables:
\renewcommand{\arraystretch}{1.3}
\newcommand{\head}[1]{\textbf{#1}}


% define new command for classes, in case we decide later on for a different style
\newcommand{\class}[1]{\emph{#1}}

\begin{document}
\newcolumntype{Y}{>{\raggedright\arraybackslash}X}

\begin{abstract}
This document describes how provenance information for astronomical datasets 
%(with the focus on observational data) 
can be modeled, stored and exchanged within 
the astronomical community in a standardized way.
We follow the definition of provenance as proposed by the W3C\footnote{\url{https://www.w3.org/TR/prov-overview/}}, i.e. that provenance is information about entities, activities, and people involved in producing a piece of data or thing, which can be used to form assessments about its quality, reliability or trustworthiness.
Such provenance information in astronomy is important to enable any scientist to trace back
the origin of a dataset (e.g. an image, spectrum, catalog or single points in a 
spectral energy distribution diagram or a light curve), learn about the people and 
organizations involved in a project and assess the quality of the dataset as well
as the usefulness of the dataset for her own scientific work.
\end{abstract}


\section*{Acknowledgments}

This document has been developed in part with support from the German
Astrophysical Virtual Observatory, funded by BMBF Bewilligungsnummer 05A14BAD and 05A08VHA.
The Provenance Working Group acknowledges support from the ASTERICS Project, funded by the European Commission (project 653477).

Thanks for fruitful discussions to (in alphabetical order):
Markus Demleitner, Harry Enke, Jochen Klar, Gerard Lemson, Markus Nullmeier
and Adrian Partl.



\section*{Conformance-related definitions}

The words ``MUST'', ``SHALL'', ``SHOULD'', ``MAY'', ``RECOMMENDED'', and
``OPTIONAL'' (in upper or lower case) used in this document are to be
interpreted as described in IETF standard, \citet{std:RFC2119}.

The \emph{Virtual Observatory (VO)} is
a general term for a collection of federated resources that can be used
to conduct astronomical research, education, and outreach.
The \href{http://www.ivoa.net}{International
Virtual Observatory Alliance (IVOA)} is a global
collaboration of separately funded projects to develop standards and
infrastructure that enable VO applications.


\section{Introduction}

In this document, we discuss a draft for an IVOA standard data model for
describing the provenance of astronomical data. 
We follow the definition of provenance as proposed by the W3C (see \url{https://www.w3.org/TR/prov-overview/}), i.e. that provenance is information about entities, activities, and people involved in producing a piece of data or thing, which can be used to form assessments about its quality, reliability or trustworthiness.

In astronomy, entities are generally datasets composed of VOTables, FITS files or database tables, or files containing logs, values (spectra, lightcurves), parameters... The activities correspond to an observation, a simulation, or processing steps (image stacking, object extraction, ...). The people involved can be individual persons (observer, publisher...) or organisations.

We note that the provenance of simulated data is already described inside the Simulation Data Model
\citep[SimDM,][]{std:SimDM}. However, the Provenance Data Model currently discussed is
sufficiently abstract that its core pattern could be applied to any kind of process using either observation or simulation data. It could also be used to describe the flow of scientific proposals from application to acceptance or the publication of scientific articles based on astronomical data.

%including extraction of data from 
%databases or even the flow of scientific proposals from application to 
%acceptance, including scheduling of the observations proposed therein.
%Provenance information could also be used to check internal processes,
%e.g., whether a proposal was approved by a person from a certain committee,
%or whether the time span between application and acceptance or rejection
%does not extend a certain period, etc... 


\subsection{Goal of the provenance model}\label{sec:goals}

The goal of the provenance data model is to describe how provenance information
can be modeled, stored and exchanged within the Virtual Observatory. Its scope
is mainly modeling of the flow of data, of the relations between data,
and of processing steps. 

Characteristics of observations such as ambient conditions and instrument characteristics can be associated to provenance information.
However, they will not be modeled here explicitly. 
They can be included in form of additional data linked to observations, or as attributes of observation processes.

In general, the model shall capture information that would enable a scientist who has no prior knowledge about a dataset to get more background information. 
This will help the scientist to decide if the dataset 
is adequate for his research goal, judge its quality and get enough information
to be able to trace back its history as far as possible. 

Provenance information may be recorded in minute detail or by using coarser
elements, depending on the intended usage and the desired level of detail
for a specific project that records provenance. 
This granularity depends on the needs of the project when implementing a system to track provenance information.

The following list is a collection of tasks which the Provenance Data Model should help to solve. They are flagged with [S] for problems which are more interesting for the end user of datasets and with [P] for tasks that are probably more important for data producers and publishers.
More specific use cases in the astronomy domain for different types of datasets and workflows along with example implementations are given in Section \ref{sec:usecases-implementations}.


\paragraphlb{A: Tracking the production history [S]}
        Find out which steps were taken to produce a dataset and list the methods/tools/software that was involved. 
        Track the history back to the raw data files/raw images, show the workflow (backwards search) or return a list of progenitor datasets.

        \noindent Examples: 
        \begin{itemize}
            \item Is an image from catalogue xxx already calibrated?
What about dark field subtraction? Were foreground stars removed? Which technique
was used?  
            \item Is the background noise of atmospheric muons still present in my neutrino data sample?  
        \end{itemize}

        We do not go so far as to consider easy reproducibility as a use case -- this would be too ambitious. But at least the 
        major steps undertaken to create a piece of data should be recoverable.


\paragraphlb{B: Attribution and contact information [S]}
        Find the people involved in the production of a dataset,
        the people/organizations/institutes that need to be cited or can be asked for more information.

        \noindent Examples: 
        \begin{itemize}
            \item I want to use an image for my own work -- who was involved in
creating it? Who do I need to cite or who can I contact to get this information?  
            \item I have a question about column xxx in a data
table. Who can I ask about that?  
            \item Who should be cited or acknowledged if I use this data in my work?
        \end{itemize}


\paragraphlb{C: Locate error sources. [S, P]}
        Find the location of possible error sources in the generation of a dataset.

        \noindent Examples:
        \begin{itemize}
            \item I found something strange in an image. Where does
the image come from? Which instrument was used, with which characteristics
etc.? Was there anything strange noted when the image was taken?  
            \item Which pipeline version was used -- the old one
with a known bug for treating bright objects or a newer version?  
            \item This light curve doesn't look quite right. How was
the photometry determined for each data point?  
        \end{itemize}


\paragraphlb{D: Quality assessment [P]}
        Judge the quality of an observation, production step or dataset.
        
        \noindent Examples:
        \begin{itemize}
            \item Since wrong calibration images may increase the
number of artifacts on an image rather than removing them, knowledge about
the calibration image set will help to assess the quality of the calibrated
image.  
        \end{itemize}
      

\paragraphlb{E: Search in structured provenance metadata [P, S]}
        This would allow one to also do a ``forward search'', i.e. locate derived datasets or outputs, e.g. finding all images produced by a certain processing step or derived from data which were taken by a given facility.
        
        \noindent Examples:
        \begin{itemize}
            \item Give me more images that were produced using the same pipeline.  
            \item Give me an overview on all images reduced with the same calibration dataset.  
            \item Are there any more images attributed to this observer?  
            \item Which images of the crab nebula are of good quality and were produced within the last 10 years by someone not from ESO or NASA?
            \item Find all datasets generated using this given algorithm for this given step of the data processing
          % add another specific use case for tracking scientific productivity?
        \end{itemize}

        This task is probably the most challenging. It also includes tracking the history of data items as in A, but we still have listed this task separately, since we may decide that we can't keep this one, but we definitely want A.

\subsection{Requirements for provenance and use cases}
\subsubsection{Requirements}\label{sec:requirements}

An IVOA provenance data model should provide solutions to the following tasks:

\paragraphlb{A: Tracking the production history}
        Find out which steps were taken to produce a dataset and list the methods/tools/software that was involved. 
        Track the history back to the raw data files/raw images, show the workflow.

        \noindent Examples: 
        \begin{itemize}
            \item Is an image from catalogue xxx already calibrated?
What about dark field subtraction? Were foreground stars removed? Which technique
was used?  
            
            \item Is the background noise of atmospheric muons still present in my neutrino data sample?  
        \end{itemize}

        We do not go so far as to consider easy reproducibility as a use case -- this would be too ambitious. But at least the 
        major steps undertaken to create a piece of data should be recoverable.

      
\paragraphlb{B: Attribution and further information}
        Find the people involved in the production of a dataset,
        the people/organizations/institutes that need to be cited or can be asked for more information.

        \noindent Examples: 
        \begin{itemize}
            \item I want to use an image for my own work -- who was involved in
creating it? Who do I need to cite or who can I contact to get this information?  
            \item I have a question about column xxx in a data
table. Who can I ask about that?  
        \end{itemize}
      

\paragraphlb{C: Aid in debugging}
        Find possible error sources.

        \noindent Examples:
        \begin{itemize}
            \item I found something strange in an image. Where does
the image come from? Which instrument was used, with which characteristics
etc.? Was there anything strange noted when the image was taken?  
            \item Which pipeline version was used -- the old one
with a known bug for treating bright objects or a newer version?  
            \item This light curve doesn't look quite right. How was
the photometry determined for each data point?  
        \end{itemize}


\paragraphlb{D: Quality assessment}
        Judge the quality of an observation, production step or dataset.
        
        \noindent Examples:
        \begin{itemize}
            \item Since wrong calibration images may increase the
number of artifacts on an image rather than removing them, knowledge about
the calibration image set will help to assess the quality of the calibrated
image.  
        \end{itemize}
      

\paragraphlb{E: Search in structured provenance metadata}
        Find all images produced by a certain processing step and similar tasks.
        
        \noindent Examples:
        \begin{itemize}
            \item Give me more images that were produced using the
same pipeline.  
            \item Give me an overview on all images reduced with the same calibration dataset.  
            \item Are there any more images attributed to this observer?  
            \item Which images of the crab nebula are of good quality and were produced within the last 10 years by someone not from ESO or NASA?  
        \end{itemize}

        This task is probably the most challenging. It also includes tracking the history of data items as in A, but we still have listed this task separately, since we may decide that we can't keep this one, but we definitely want A.


\subsubsection{More specific use cases}
More specific use cases with example serialisations for different types of astronomical datasets are given in Section \ref{sec:usecases-implementations}.

\subsection{Role within the VO Architecture}
\TODO{Will be inserted later.}
% Skipping this for now. Let's not let this draft look more official than it 
% currently is.

%\begin{figure}
%\centering
%\includegraphics[width=0.9\textwidth]{archdiag.png}
%\caption{Architecture diagram for this document}
%\label{fig:archdiag}
%\end{figure}

%Fig.~\ref{fig:archdiag} shows the role this document plays within the
%IVOA architecture \citep{note:VOARCH}.

\subsection{Previous efforts}

Outside of the astronomical community, the Provenance Challenge series (2006 -- 2010), a community effort to achieve inter-operability between different representations of provenance in scientific workflows, resulted in the Open Provenance Model (\cite{moreau2010}). 
Later, the W3C Provenance Working Group was founded and released the W3C Provenance Data Model as Recommendation in 2013 (\cite{std:W3CProvDM}). 
OPM was designed to be applicable to anything, scientific data as well as cars or immaterial things like decisions. With the W3C model, this becomes more focused on the web.  Nevertheless, the core concepts are still in principle the same in both models and very general, so they can be applied to astronomical datasets and workflows as well. 
The W3C model was taken up by a larger number of applications and tools than OPM, we are therefore basing our modeling efforts on the W3C Provenance data model, making it less abstract and more specific, or extending it where necessary. 


The W3C model even already specifies PROV-DM Extensibility points (section 6 in \cite{std:W3CProvDM}) for extending the core model. This allows to specify additional roles and types to each entity, agent or relation using the attributes \texttt{prov:type} and \texttt{prov:role}.
By specifying the allowed values for the IVOA model, we could adjust the model to our needs while still being compliant to W3C.




\section{The provenance data model}

\subsection{Overview and class diagram}
\label{sec:overview}


\begin{figure}[hbt]
\centering
\includegraphics[width=1.0\textwidth]{PROV_Fig3.png}
\caption[Overview class diagram of the IVOA Provenance Data Model]{Overview class diagram of the IVOA Provenance Data Model. The core part in yellow is based on W3C PROV definitions where relations are shown in grey. It is extended by a description part (orange), specific types of entities (red) and an \class{ActivityConfiguration} package (green). A full diagram with attributes is shown in Section~\ref{sec:fulldiagram}, Figure~\ref{fig:fulldiagram}}
\label{fig:overview}
\end{figure}

The IVOA Provenance DM is based on the the PROV-DM recommendation \citep{std:W3CProvDM} of the World Wide Web Consortium (W3C), that provides the core elements of the model (see Sections~\ref{sec:ent_act} to~\ref{sec:agent+relations}). 
In the VO context, the provenance of something is thus a sequence of activities using and generating entities run by agents.

The model also includes description classes (see Section~\ref{sec:descriptions}) to provide information common to several elements; Specific types of \class{Entity} classes commonly used in astronomy (see Section~\ref{sec:spec_entities}); and an \class{ActivityConfiguration} package (see Section~\ref{sec:configuration}).

The IVOA Provenance DM is a class data model that follows the VO-DML designing rules \citep{2018ivoa.spec.0910L}. It is represented as a UML class diagram: an overview diagram is shown in Figure~\ref{fig:overview}, and a full diagram with attributes is shown in Appendix~\ref{sec:fulldiagram}, Figure~\ref{fig:fulldiagram}.


\subsection{Entity and Activity classes}
\label{sec:ent_act}

The core classes and relations of the IVOA Provenance DM are presented in Figure~\ref{fig:coreclasses}.
Traceability (see goal A in Section~\ref{sec:goals}) is enabled by chaining entities and activities, which are the building blocks of the history graph.


\begin{figure}[ht]
\centering
\includegraphics[width=1.0\textwidth]{PROV_Fig4.png}
\caption[Core classes and relations]{Core classes and relations. Attributes for these classes are detailed in tables found in Sections~\ref{sec:ent_act} to~\ref{sec:agent+relations}.}
\label{fig:coreclasses}
\end{figure}



\subsubsection{Entity and Collection classes}
\label{sec:Entity}

An \textbf{entity} is a physical, digital, conceptual, or other kind of thing with some fixed aspects (W3C PROV-DM \href{https://www.w3.org/TR/prov-dm/#term-entity}{\S5.1.1}). 

The \class{Entity} class in the model has the attributes given in Table \ref{tab:entity}.

Entities in astronomy are usually astronomical or astrophysical datasets in the form of images, tables, numbers, etc. But they can also be log files, files containing system information, any input or output value, environment variables, ambient conditions, or, in a wider sense, observation proposals, scientific articles, or manuals and other documents. 
Though the focus is on digital entities in this document, entities can also refer to physical entities that may be linked to digital entities, such as e.g., tools, instruments, detectors, photographic plates.


\begin{table}[ht]
\small
\tymax  0.5\textwidth
\textbf{\normalsize Entity}\vspace{0.25em}\\
\begin{tabulary}{1.0\textwidth}{llL}
\toprule
\head{Attribute} & \head{Data type} & \head{Description}\\
\midrule
\textbf{id} & string & a unique identifier for this entity\\
name        & string & a human-readable name for the entity\\
location    & string & a path or spatial coordinates, e.g., a URL/URI, latitude-longitude coordinates on Earth, the name of a place.\\
generatedAtTime  & datetime & date and time at which the entity was created (e.g., timestamp of a file)\\
invalidatedAtTime  & datetime & date and time of invalidation of the entity. After that date, the entity is no longer available for any use.\\
comment  &  string & text containing specific comments on the entity\\
\bottomrule
\end{tabulary}
\caption[Attributes of the \class{Entity} class]{Attributes of the \class{Entity} class. Attributes in \textbf{bold} are mandatory and must not be null.
}\label{tab:entity}
\end{table}


A \textbf{collection} is an entity that provides a structure to some constituents that must themselves be entities (W3C PROV-DM \href{https://www.w3.org/TR/prov-dm/#term-collection}{\S5.6.1}). These constituents are said to be members of the collections. They are connected in the model with a \class{hadMember} relation.


\subsubsection{Activity class}
\label{sec:activity}

An \textbf{activity} is something that occurs over a period of time and acts upon or with entities; it may include consuming, processing, transforming, modifying, relocating, using, or generating entities (W3C PROV-DM \href{https://www.w3.org/TR/prov-dm/#term-Activity}{\S5.1.2}). 

The \class{Activity} class in the model has the attributes given in Table \ref{tab:activity}.

Activities in astronomy include all steps from obtaining data to the reduction
of  images and production of new datasets, such as image calibration, bias
subtraction, image stacking, light curve generation from a number of
observations, radial velocity determination from spectra, post-processing steps
of simulations, etc.


\begin{table}[ht]
\small
\tymax  0.5\textwidth
\textbf{\normalsize Activity}\vspace{0.25em}\\
\begin{tabulary}{1.0\textwidth}{llL}
\toprule
\head{Attribute}  & \head{Data type} & \head{Description}\\
\midrule
\textbf{id}  & string & a unique id for this activity\\
name         & string & a human-readable name (to be displayed by clients)\\
startTime    & datetime & start of an activity\\
endTime      & datetime & end of an activity\\
comment      & string & text containing specific comments on the activity\\
\bottomrule
\end{tabulary}
\caption[Attributes of the \class{Activity} class.]{Attributes of the \class{Activity} class. Attributes in \textbf{bold} are mandatory and must not be null.}\label{tab:activity}
%, references are indicated with an arrow ($\rightarrow$).}
\end{table}


\subsection{Entity-Activity relations}
\label{sec:entity-activity-relations}

Each entity is usually a result of an activity, expressed by a link from the entity to its generating activity, and can be used as input for (many) other activities.
Thus the information on whether data are used as input or were produced as output of some activity is given by the \emph{relations} between activities and entities.
Tracking those relations answers one of the main objectives of the model (see goal A in Section~\ref{sec:goals}).


\subsubsection{Used class}

\textbf{Usage} is the beginning of utilizing an entity by an activity. Before usage, the activity had not begun to utilize this entity and could not have been affected by the entity (W3C PROV-DM \href{https://www.w3.org/TR/prov-dm/#term-Usage}{\S5.1.4}).

Usage is implemented in the model by a class \class{Used} that connects \class{Activity} to \class{Entity} and contains the attributes in Table~\ref{tab:used}.

For example, an activity ``calibration'' used entities with the roles ``calibration data'' and ``raw images''.

\begin{table}[ht]
\small
\tymax  0.5\textwidth
\textbf{\normalsize Used}\vspace{0.25em}\\
\begin{tabulary}{1.0\textwidth}{llL}
\toprule
\head{Attribute} & \head{Data type} & \head{Description}\\
\midrule
role  & string   & function of the entity with respect to the activity\\
time  & datetime & time at which the usage of an entity started\\
\bottomrule
\end{tabulary}
\caption[Attributes of the \class{Used} relation class]{Attributes of the \class{Used} relation class.}
\label{tab:used}
\end{table}

The \attribute{time} of the usage can be specified, and must be between the \attribute{startTime} and the \attribute{endTime} of the corresponding activity.

The \class{Used} class is closely coupled to the \class{Activity} by a composition (see \ref{sect:Composition}). 
Any given entity can be used by more than one activity.


\subsubsection{WasGeneratedBy class}

\textbf{Generation} is the completion of production of a new entity by an activity. This entity did not exist before generation and becomes available for usage after this generation (W3C PROV-DM \href{https://www.w3.org/TR/prov-dm/#term-Generation}{\S5.1.3}).
        
Generation is implemented in the model by a class \class{WasGeneratedBy} that connects \class{Entity} to \class{Activity} and contains the attributes in Table~\ref{tab:wasgeneratedby}.

For example, the entity ``raw\_image.fits'' was generated by the activity ``observation'' with the role ``raw image''.

\begin{table}[ht]
\small
\tymax  0.5\textwidth
\textbf{\normalsize WasGeneratedBy}\vspace{0.25em}\\
\begin{tabulary}{1.0\textwidth}{llL}
\toprule
\head{Attribute} & \head{Data type} & \head{Description}\\
\midrule
role   &  string   &  function of the entity with respect to the activity\\
%time & prov:time & datetime & time at which the generation of an entity is finished\\
\bottomrule
\end{tabulary}
\caption[Attributes of the \class{WasGeneratedBy} relation class]{Attributes of the \class{WasGeneratedBy} relation class.}
\label{tab:wasgeneratedby}
\end{table}

As the \class{Entity} class has an attribute \attribute{generatedAtTime}, there is no additional time attribute in this relation.

The \class{WasGeneratedBy} relation is closely coupled with the \class{Entity} via a composition (see \ref{sect:Composition}). 
An entity can be generated by only one activity, so the multiplicity is 1 or 0 between \class{Entity} and \class{WasGeneratedBy}.


\subsubsection{Roles in Entity-Activity relations}
\label{sec:roles}

The \attribute{role} of an entity within an activity should be provided.
Roles in \class{Entity}-\class{Activity} relations are free text attributes.

The \attribute{role} cannot be an attribute of the \class{Entity} class, since the same entity (e.g., a specific file containing an image) may play different roles with different activities.

In some cases the role is mandatory to distinguish two input entities. For example, an activity for dark-frame subtraction requires two input images. But it is very important to know which of the images is the raw image and which one fulfils the role of dark-frame.

Several entities may play the same role for an activity. For example, many image entities may be used as science-ready images for an image stacking process.



\subsubsection{WasDerivedFrom relation}

A \textbf{derivation} is a transformation of an entity into another, an update of an entity resulting in a new one, or the construction of a new entity based on a pre-existing entity (W3C PROV-DM \href{https://www.w3.org/TR/prov-dm/#term-Derivation}{\S5.2.1}).

Derivation is a relation \class{wasDerivedFrom} in the model, that connects an instance of \class{Entity} to another instance.

For example, the entity ``calibrated\_image.fits'' was derived from the entity ``raw\_image.fits''

This relation makes it possible to visualize independently the flow of entities, e.g., a dataflow. It does not need a priori a specific class or table in an implementation, but it provides a way to expose partial information that follows the general chain \class{WasGeneratedBy}-\class{Activity}-\class{Used} where the activity may be an empty instance because it is unknown or irrelevant.


\subsubsection{WasInformedBy relation}

\textbf{Communication} is the exchange of information (some unspecified entity) by two activities, one activity using some entity generated by the other (W3C PROV-DM \href{https://www.w3.org/TR/prov-dm/#term-Communication}{\S5.1.5}).

Communication is a relation \class{wasInformedBy} in the model, that connects an instance of \class{Activity} to another instance.

For example, the activity ``calibration'' was informed by the activity ``pipeline''.

This relation makes it possible to visualize independently the flow of activities as they occurred, which may be the result of the execution of a workflow. It does not need a priori a specific class or table in an implementation, but it provides a way to expose partial information that follows the general chain \class{Used}-\class{Entity}-\class{WasGeneratedBy} where the entity may be an empty instance because it is unknown or irrelevant.


\subsection{Agent and relations to Agent}
\label{sec:agent+relations}

A contact information is needed in case more information about a certain activity or entity is required, but also in order to know who was involved and to fulfil the Acknowledgement objective (see goal B in Section~\ref{sec:goals}).


\subsubsection{Agent class}
\label{sec:agent}

An \textbf{agent} is something that bears some form of responsibility for an activity taking place or for the existence of an entity (W3C PROV-DM \href{https://www.w3.org/TR/prov-dm/#term-agent}{\S5.3.1}).

The \class{Agent} class in the model has the attributes given in Table \ref{tab:agent}. 

An Agent is generally someone who pressed a button, ran a script, performed the observation or published a dataset. The agent can be a single person, a group of persons, a project or an institute (the vocabulary for agent types is given in Table~\ref{tab:agent-types}).

It is recommended to use organizational agents and agents with generic contacts.


\begin{table}[ht]
\small
\tymax  0.5\textwidth
\textbf{\normalsize Agent}\vspace{0.25em}\\
\begin{tabulary}{1.0\textwidth}{llL}
\toprule
\head{Attribute} & \head{Data type} & \head{Description}\\
\midrule
\textbf{id}    & string & unique identifier for an agent\\
\textbf{name}  & string & a common name for this agent; e.g., first name and last name; project name,  pipeline team, data center.\\
type        & AgentType & type of the agent as given in Table~\ref{tab:agent-types}\\
comment     & string & text containing specific comments on the agent\\
email       & string & contact email of the agent\\
affiliation & string & affiliation of the agent\\
phone       & string & phone number\\
address     & string & address of the agent\\
url         & anyURI & reference URL to the agent\\
\bottomrule
\end{tabulary}
\caption[Attributes of the \class{Agent} class]{Attributes of the \class{Agent} class. Attributes in \textbf{bold} are mandatory and must not be null.}
\label{tab:agent}
\end{table}

\begin{table}[ht]
\small
\tymax  0.5\textwidth
\textbf{\normalsize AgentType}\vspace{0.25em}\\
\begin{tabulary}{1.0\textwidth}{lp{8cm}}
\toprule
\head{Type} &\head{Description} \\
\midrule
Person        & person agents are people\\
Organization  & a social or legal institution, e.g., an institute, a consortium, a project\\
SoftwareAgent & running software, e.g., a cron job or a trigger \\
\bottomrule
\end{tabulary}
\caption[Enumeration of Agent types.]{Enumeration of Agent types.}
\label{tab:agent-types}
\end{table}

% 2018-12 commented
%A definition of organizations is given in the 
%IVOA Recommendation on Resource Metadata \citep{std:ResourceMeta}, hereafter 
%referred to as RM: ``An organization is [a] specific type of resource that 
%brings people together to pursue participation in VO applications.''
%It also specifies further that scientific projects can be considered 
%as organizations on a finer level:
%``At a high level, an organization could be a university, observatory, or government
%agency. At a finer level, it could be a specific scientific project, space mission,
%or individual researcher. A provider is an organization that makes data and/or services
%available to users over the network.''

For each agent a \attribute{name} must be specified. 
Other attributes can help locate or contact the agent (\attribute{email}, \attribute{affiliation}, \attribute{phone}, \attribute{address}).
Not every project will need them; e.g. an advanced system may use permanent identifiers (ORCIDs, identities in federations, etc) to identify agents, and retrieve their properties from an external system instead.

There can be more than one agent for each activity and one agent can be responsible for more than one activity or entity, using the relations defined in the following sections.


\subsubsection{WasAssociatedWith class}

An activity \textbf{association} is an assignment of responsibility to an agent for an activity, indicating that the agent had a role in the activity (W3C PROV-DM \href{https://www.w3.org/TR/prov-dm/#term-Association}{\S5.3.3}).

Association is implemented in the model by a class \class{WasAssociatedWith} that connects \class{Activity} to \class{Agent} and contains the attributes in Table~\ref{tab:wasassociatedwith}.

For example, the agent ``Max Smith'' was associated with the activity ``observation'' with the role ``Observer''.

\begin{table}[ht]
\small
\tymax  0.5\textwidth
\textbf{\normalsize WasAssociatedWith}\vspace{0.25em}\\
\begin{tabulary}{1.0\textwidth}{llL}
\toprule
\head{Attribute} & \head{Data type} & \head{Description}\\
\midrule
role & string   & function of the agent with respect to the activity, see Section~\ref{sec:agent_roles} \\
\bottomrule
\end{tabulary}
\caption[Attributes of \class{WasAssociatedWith} relation class]{Attributes of \class{WasAssociatedWith} relation class.}
\label{tab:wasassociatedwith}
\end{table}


\subsubsection{WasAttributedTo class}

\textbf{Attribution} is the ascribing of an entity to an agent. When an entity is attributed to an agent, this entity was generated by some unspecified activity that in turn was associated to the agent. Thus, this relation is generally useful when the activity is not known, or irrelevant (W3C PROV-DM \href{https://www.w3.org/TR/prov-dm/#term-attribution}{\S5.3.2}). 

Attribution is implemented in the model by a class \class{WasAttributedTo} that connects \class{Entity} to \class{Agent} and contains the attributes in Table~\ref{tab:wasattributedto}.

For example, the entity ``science\_image.fits'' was attributed to the agent ``observatory''.


\begin{table}[ht]
\small
\tymax  0.5\textwidth
\textbf{\normalsize WasAttributedTo}\vspace{0.25em}\\
\begin{tabulary}{1.0\textwidth}{llL}
\toprule
\head{Attribute} & \head{Data type} & \head{Description}\\
\midrule
role & string   & function of the agent with respect to the entity, see Section~\ref{sec:agent_roles} \\
\bottomrule
\end{tabulary}
\caption[Attributes of \class{WasAttributedTo} relation class]{Attributes of \class{WasAttributedTo} relation class.}
\label{tab:wasattributedto}
\end{table}


\subsubsection{Agent roles}
\label{sec:agent_roles}

Agents may play a specific role with respect to an activity or an entity. 
The \attribute{role} attribute should be specified whenever it is known.

Roles in relations to \class{Agent} are free text attributes, but if one of the terms in Table \ref{tab:agent-roles} applies, it should be used.

% DataCite roles, see https://schema.datacite.org/meta/kernel-4.2/doc/DataCite-MetadataKernel_v4.2.pdf
% ContactPerson DataCollector DataCurator DataManager Distributor Editor HostingInstitution Producer ProjectLeader ProjectManager ProjectMember RegistrationAgency RegistrationAuthority RelatedPerson Researcher ResearchGroup RightsHolder Sponsor Supervisor WorkPackageLeader Other

\begin{table}[ht]
\small
\tymax  0.5\textwidth
\textbf{\normalsize Agent roles}\vspace{0.25em}\\
\begin{tabulary}{1.0\textwidth}{lp{8cm}}
\toprule
\head{Role} & \head{Description} \\
\midrule
Author       & agent at the origin of a written entity (e.g., article, document, proposal) \\
Contributor* & agent responsible for making contributions to an entity or an activity \\
Coordinator  & agent leading the organisation of an activity \\
Creator*     & agent primarily responsible for creating an entity or an activity \\
Curator      & agent responsible for the legacy aspects of an entity \\
Editor       & agent that edited and validated the content of an entity \\
Funder       & agent that provided financial support for an activity or an entity \\
Investigator & agent responsible for the scientific goals of an activity \\
Observer     & agent responsible for an observation activity or the result of an observation \\
Operator     & agent in charge of performing an activity or using an entity \\
Provider*    & agent that effectively delivered an entity or a service \\
Publisher*   & agent that certified and was responsible for making an entity available to the public \\
\bottomrule
\end{tabulary}
\caption[Terms applicable as agent roles.]{Terms applicable as agent roles. Terms marked with an * are also found in other IVOA documents \citep[e.g.,][]{2007ivoa.spec.0302H,2017ivoa.spec.0524G}}
\label{tab:agent-roles}
\end{table}




\subsection{Description classes}
\label{sec:descriptions}

In the domain of astronomy, certain processes and steps are repeated over and over again, maybe using a different configuration and within a different context. 
We therefore separate the descriptions of activities from the actual processes and introduce an \class{ActivityDescription} class (Section~\ref{sec:activity_desc}). 
Likewise, we also apply the same pattern for \class{Entity} and add an \class{EntityDescription} class (Section~\ref{sec:entity_desc}). 

Defining such descriptions allows them to be predefined and reused, which is less redundant when exposing the provenance of a series of tasks of the same type. 
Providing detailed descriptions to activities and entities help assess the quality and reliability of the processes executed (see goal C in Section~\ref{sec:goals}).

Figure~\ref{fig:classdiagram_descriptions} shows the class diagram part focused on the description classes. 

\begin{figure}[ht]
\centering
\includegraphics[width=1.0\textwidth]{PROV_Fig5.png}
\caption[Partial class diagram focused on description classes.]{Partial class diagram focused on description classes.}
\label{fig:classdiagram_descriptions}
\end{figure}


\subsubsection{ActivityDescription class}
\label{sec:activity_desc}


\begin{table}[ht]
\small
\tymax  0.5\textwidth
\textbf{\normalsize ActivityDescription}\vspace{0.25em}\\
\begin{tabulary}{1.0\textwidth}{llL}
\toprule
\head{Attribute} &  \head{Data type} & \head{Description}\\
\midrule
\textbf{name}         & string & a human-readable name\\
version      & string & a version number, if applicable (e.g., for the code used)\\
description  & string & additional free text describing how the activity works internally\\
docurl       & anyURI & link to further documentation on this activity, e.g., a
paper, the source code in a version control system etc.\\
type        & string & type of the activity\\
subtype     & string & more specific subtype of the activity\\
\bottomrule
\end{tabulary}
\caption[Attributes of the \class{ActivityDescription} class]{Attributes of the \class{ActivityDescription} class. Attributes in \textbf{bold} are mandatory and must not be null.
}\label{tab:activitydescription}
\end{table}


The information necessary to describe how an activity works internally are stored in \class{ActivityDescription} objects.

\class{ActivityDescription} is directly attached to \class{Activity} and can thus be seen as a list of attributes that can be known before an \class{Activity} instance is created.

There must be exactly zero or one \class{ActivityDescription} instance per activity.
If an activity is linked to an \class{ActivityDescription} instance, \class{Used}/\class{WasGeneratedBy}/\class{Entity} objects bound to this activity must refer to the description elements composing the \class{ActivityDescription}.

The activity \attribute{type} is a free text attribute, but if one of the terms in Table \ref{tab:activitydescription-types} applies, it should be used.
The activity \attribute{subtype} is a free text attribute to be used internally by the project that defined \class{ActivityDescription} instances (e.g., mosaicing, denoising, photometric calibration, cross correlation).


\begin{table}[ht]
\small
\tymax  0.5\textwidth
\textbf{\normalsize ActivityDescription types}\vspace{0.25em}\\
\begin{tabulary}{1.0\textwidth}{lp{8cm}}
\toprule
\head{Type} & \head{Description} \\
\midrule
Observation    & active acquisition of information on a phenomenon\\
Simulation     & generation of data through a computational process\\
Reduction      & transformation of digital information into a corrected, ordered, and simplified form\\
Calibration    & transformation and comparison of measurement values with respect to a calibration standard of known accuracy\\
Reconstruction & estimation of physical properties using indirect information\\
Selection      & application of filters or criteria to select partial information\\
Analysis       & process of inspecting, cleaning, transforming, and modeling data with the goal of discovering useful information, informing conclusions, and supporting decision-making\\
\bottomrule
\end{tabulary}
\caption[Terms applicable as activity types.]{Terms applicable as activity types.}
\label{tab:activitydescription-types}
\end{table}



\subsubsection{EntityDescription class}
\label{sec:entity_desc}


\begin{table}[ht]
\small
\tymax  0.5\textwidth
\textbf{\normalsize EntityDescription}\vspace{0.25em}\\
\begin{tabulary}{\textwidth}{llL}
\toprule
\head{Attribute} & \head{Data type} & \head{Description}\\
\midrule
\textbf{name}       & string & a human-readable name for the entity description\\
description  & string & a descriptive text for this kind of entity\\
docurl   & anyURI & link to more documentation\\
type      & string & type of the entity\\
\bottomrule
\end{tabulary}
\caption[Attributes of the \class{EntityDescription} class]{Attributes of the \class{EntityDescription} class. Attributes in \textbf{bold} are mandatory and must not be null.
}\label{tab:entitydescription}
\end{table}


The \class{EntityDescription} class is meant to store descriptive information for different categories of entities. It contains information that is known before an \class{Entity} instance is created. The \class{EntityDescription} general attributes are summarized in Table~\ref{tab:entitydescription}.

For example, a specific category of entities in a project may be defined in details in a document or on a webpage (e.g., a CTA DL3 file, a CCD device, a photographic plate).

The entity \attribute{type} is a free text attribute, that contains the general category of the entity, e.g., if it is data, a document, a vizualization, a device.

The \class{EntityDescription} class should not contain information about the usage of the data, in particular, it generally tells nothing about them being used as input or generated as output. This kind of information should be provided by the relations (and their descriptions) between activities and entities (see Sections~\ref{sec:entity-activity-relations} and \ref{sec:use_gen_desc}).


\subsubsection{UsageDescription and GenerationDescription classes}
\label{sec:use_gen_desc}

\begin{table}[h!]
\small
\tymax  0.5\textwidth
\textbf{\normalsize UsageDescription}\vspace{0.25em}\\
\begin{tabulary}{1.0\textwidth}{llL}
\toprule
\head{Attribute} &  \head{Data type} & \head{Description}\\
\midrule
\textbf{role} & string   & function of the entity with respect to the activity \\
description  & string & a descriptive text for this kind of usage \\
type    & string   & type of relation, see Section~\ref{sec:ugtypes} \\
multiplicity & string & Number of expected input entities to be used with the given role. The multiplicity syntax is similar to that of VO-DML (\citealt{2018ivoa.spec.0910L}, \S4.19) in the form `minOccurs..maxOccurs'' or a single value if minOccurs and maxOccurs are identical, e.g., ``1'' for one item, ``*'' for unbounded or ``3..*'' for unbounded with at least 3 items. \\
\bottomrule
\end{tabulary}
\caption[Attributes of the \class{UsageDescription} class]{Attributes of the \class{UsageDescription} class. Attributes in \textbf{bold} are mandatory and must not be null.}
\label{tab:usagedescription}
\end{table}


\begin{table}[h!]
\small
\tymax  0.5\textwidth
\textbf{\normalsize GenerationDescription}\vspace{0.25em}\\
\begin{tabulary}{1.0\textwidth}{llL}
\toprule
\head{Attribute} & \head{Data type} & \head{Description}\\
\midrule
\textbf{role} & string & function of the entity with respect to the activity \\
description  & string & a descriptive text for this kind of generation \\
type & string   & type of relation, see section \ref{sec:ugtypes} \\
multiplicity & string & Number of expected output entities that will be generated with the given role. The multiplicity syntax is similar to that of VO-DML (\citealt{2018ivoa.spec.0910L}, \S4.19) in the form `minOccurs..maxOccurs'' or a single value if minOccurs and maxOccurs are identical, e.g., ``1'' for one item, ``*'' for unbounded or ``3..*'' for unbounded with at least 3 items. \\
\bottomrule
\end{tabulary}
\caption[Attributes of the \class{GenerationDescription} class]{Attributes of the \class{GenerationDescription} class. Attributes in \textbf{bold} are mandatory and must not be null.}
\label{tab:wasgeneratedbydescription}
\end{table}


In order to describe more precisely an activity, the expected inputs and outputs of this activity should be specified.

We introduce the \class{UsageDescription} and the \class{GenerationDescription} classes, that are meant to store the information about the usage or generation of entities that is known before an activity instance is executed, i.e.~what we expect to store in the \class{Used} and \class{WasGeneratedBy} relations (see \ref{sec:entity-activity-relations}).
Instances of \class{Used} (respectively \class{WasGeneratedBy}) may thus point to an instance of \class{UsageDescription} (respectively \class{GenerationDescription}).

If a \class{UsageDescription} (respectively \class{GenerationDescription}) instance is defined, the \attribute{role} attribute of the related \class{Used} (respectively \class{WasGeneratedBy}) instances must match the \attribute{role} attribute of this \class{UsageDescription} (respectively \class{GenerationDescription}) instance.

A \attribute{multiplicity} attribute should be specified to indicate the number of entities expected to share the same role for a given \class{ActivityDescription} instance, e.g., in the case of the stacking of images, several images are expected with the same input role (\attribute{multiplicity=*}).

When related to the \class{UsageDescription} or \class{GenerationDescription}, the attributes of \class{EntityDescription} (see Section~\ref{sec:entity_desc}) help to describe the category of entities expected as an input or an output in an activity.
For example: if the input bias files are expected to be in FITS format, the \class{UsageDescription} object would have a relation to a \class{DatasetDescription} object with \attribute{contentType}=`application/fits'' (see Section~\ref{sec:dataset_entity} for this specific type of entity).


\subsubsection{Types of Usage and Generation}
\label{sec:ugtypes}

The typing of those relations is particularly needed to enable quality assessment and identification of error sources in the process (see goals C and D in Section \ref{sec:goals}), so as to facilitate the exploration of provenance information. 

The type of usage or generation is a free text attribute, but if one of the terms in Table \ref{tab:usage-generation-types} applies, it should be used.

\begin{table}[ht]
\small
\tymax  0.5\textwidth
\begin{tabulary}{1.0\textwidth}{Lp{8cm}}
\toprule
\head{Type} & \head{Description} \\
\midrule
Main           & main input or output entities of the activity, i.e.~strictly necessary, and the primary objective of the activity\\
Calibration    & usage of an entity to calibrate another entity\\
Preview        & generation of a quick representation of an entity\\
Setup          & usage of an entity as configuration information, see also Section~\ref{sec:configurationpackage}\\
Quality        & generation of information that helps to assess the quality of the activity results, e.g., errors, warnings, flags, percentage of overexposed pixels\\
Log            & generation of logging information\\
Context        & contextual information that influences the activity, but for which there are no or little control at the moment of its execution, examples: temperature, wind, conditions of observation, execution platform, operating system, instrumental context\\
\bottomrule
\end{tabulary}
\caption[Terms applicable as usage or generation type.]{Terms applicable as usage or generation type.}
\label{tab:usage-generation-types}
\end{table}

The type ``Main'' indicates the main input and output entities of an activity. It should help to provide the minimum relevant data flow to the initial entity or activity, i.e.~to find the most relevant progenitors.


\subsection{Specific types of Entity classes}
\label{sec:spec_entities}

\class{Entity} and \class{EntityDescription} classes carry the minimum metadata that can apply to any kind of entity without specifying the nature or the structure of the content of the entity. 
In some cases, the structure of the content is relevant information to assess the usefulness of the entity, in particular for datasets.
In some other cases, the content itself of an entity is relevant information to assess the usefulness of the related entities or activities. Such content must then be exposed as properly described values.

In astronomy and the VO, we thus define two main types of entity classes:

\begin{itemize}
    \item \textbf{Dataset}: a dataset is a resource which encodes data in a defined structure. It is generally a file or a set of files which are considered to be a single deliverable. The content may be e.g., a cube, an image, a table, a list.
    \item \textbf{Value}: a value is an atomic piece of data with a given value type (e.g., a data type such as boolean, integer, real, string).
\end{itemize}

\begin{figure}[ht]
\centering
\includegraphics[width=1.0\textwidth]{PROV_Fig6.png}
\caption[Partial class diagram focused on specific types of \class{Entity} classes.]{Partial class diagram focused on Specific types of \class{Entity} classes.}
\label{fig:classdiagram_entityclasses}
\end{figure}

As shown in Figure~\ref{fig:classdiagram_entityclasses}, the entity description classes for both \class{ValueEntity} and \class{DatasetEntity} are subsetted respectively as \class{ValueDescription} and \class{DatasetDescription}.

We anticipate that more specific categories of entities can be defined by the projects (for example, a device, a document, a vizualization). The \attribute{type} attribute of the \class{EntityDescription} class should be used to differentiate the different categories of entities.


\subsubsection{DatasetEntity and DatasetDescription classes}
\label{sec:dataset_entity}

The handling of datasets is implemented in the model by a \class{DatasetEntity} class. A corresponding \class{DatasetDescription} class contains a \attribute{contentType} attribute that must not be null (see Table~\ref{tab:datasetdescription}).

The \attribute{contentType} indicates the MIME-type or format of a dataset, or a more precise structure, following the definition of the attribute \attribute{access\_format} defined in ObsCoreDM (\citet{2017ivoa.spec.0509L}, Section 4.7).

\begin{table}[ht]
\small
\tymax  0.5\textwidth
\textbf{\normalsize DatasetDescription}\vspace{0.25em}\\
\begin{tabulary}{1.0\textwidth}{llL}
\toprule
\head{Attribute} &  \head{Data type} & \head{Description}\\
\midrule
\textbf{contentType}  & string  & format of the dataset, MIME type when applicable \\
\bottomrule
\end{tabulary}
\caption[Attributes of the \class{DatasetDescription} class]{Attributes of the  \class{DatasetDescription} class. The class also inherits the attributes of \class{EntityDescription} listed in Table \ref{tab:entitydescription}. Attributes in \textbf{bold} are mandatory and must not be null.}
\label{tab:datasetdescription}
\end{table}


\subsubsection{ValueEntity and ValueDescription classes}

The handling of values is implemented in the model by a \class{ValueEntity} class that contains a \attribute{value} attribute. A corresponding \class{ValueDescription} class contains attributes commonly used in the VO to qualify values. Those attributes are listed in Table~\ref{tab:valuedescription}.

\begin{table}[ht]
\small
\tymax  0.5\textwidth
\textbf{\normalsize ValueEntity}\vspace{0.25em}\\
\begin{tabulary}{1.0\textwidth}{llL}
\toprule
\head{Attribute} &  \head{Data type} & \head{Description}\\
\midrule
\textbf{value}  & string  & the value of the entity. If a corresponding \class{ValueDescription}.\attribute{valueType} attribute is set, the value string can be interpreted by this \attribute{valueType}. \\
\bottomrule
\end{tabulary}
\caption[Attributes of the \class{ValueEntity} class]{Attributes of the  \class{ValueEntity} class. The class also inherits the attributes of \class{EntityDescription} listed in Table \ref{tab:entitydescription}. Attributes in \textbf{bold} are mandatory and must not be null.}
\label{tab:valueentity}
\end{table}

\begin{table}[ht]
\small
\tymax  0.5\textwidth
\textbf{\normalsize ValueDescription}\vspace{0.25em}\\
\begin{tabulary}{1.0\textwidth}{p{2cm}LL}
\toprule
\head{Attribute} &  \head{Data type} & \head{Description}\\
\midrule
\textbf{valueType} & VotableFieldFormat & description of a value from a combination of \attribute{datatype}, \attribute{arraysize} and \attribute{xtype} following VOTable 1.3 \citep[][, \S4.1]{2013ivoa.spec.0920O} \\
unit        & Unit & VO unit, see \ref{sect:Units} and \citet{2014ivoa.spec.0523D} for recommended unit representation \\
ucd         & string  & Unified Content Descriptor, supplying a standardized classification of the physical quantity, see \citet{2018ivoa.spec.0527M}\\
utype       & string  & Utype, meant to express the role of the value in the context of an external data model, see \citet{note:utypeusage} \\
\bottomrule
\end{tabulary}
\caption[Attributes of the \class{ValueDescription} class]{Attributes of the \class{ValueDescription} class. The class also inherits the attributes of \class{EntityDescription} listed in Table \ref{tab:entitydescription}. Attributes in \textbf{bold} are mandatory and must not be null.}
\label{tab:valuedescription}
\end{table}



\subsection{Activity configuration}
\label{sec:configuration}

Configuring an activity is the way to set parameters so that the activity occurs in the desired conditions.

In some cases developed in Section~\ref{sec:goals} (goals C and D in particular), configuration information is relevant to assess the quality and reliability of an activity or an entity, and to identify the location of configuration errors in a processing. It also facilitates the re-execution of an activity (reproducibility).

Configuration information may be carried by entities using the core features, where an entity (e.g., \class{ValueEntity} and \class{DatasetEntity} instances) is referenced in \class{Used} relations with a given \attribute{role} and \attribute{type}=“Setup”. With this solution, the configuration information is independent from the activity and can be generated and used as any entity.

The data model also provides a specialized \class{ActivityConfiguration} package to directly attach configuration information to an activity. This package is composed of a \class{WasConfiguredBy} relation connecting  \class{Parameter} and \class{ConfigFile} classes with the \class{Activity} class (see~\ref{sec:configurationpackage}). With this solution the configuration information is independent from the entities, and seen as part of the activity.


\begin{figure}[hbt]
\centering
\includegraphics[width=1.0\textwidth]{PROV_Fig7.png}
% Mireille: updated the diagram file for the last version with the proper cardinalities for Parameter and ConfigFile
\caption[Partial class diagram focused on the \class{ActivityConfiguration} package.]{Partial class diagram focused on the \class{ActivityConfiguration} package. The \class{Parameter} and \class{ConfigFile} classes provide configuration information for an \class{Activity} instance. The right side of the diagram shows the descriptions, where an \class{ActivityDescription} class is bound with the \class{ParameterDescription} and \class{ConfigFileDescription} classes.}
\label{fig:activityconfig}
\end{figure}


\subsubsection{Overview of the ActivityConfiguration package} \label{sec:configurationpackage}

As shown in Figure \ref{fig:activityconfig} the \class{ActivityConfiguration} package contains two classes for the execution side: \class{Parameter} and \class{ConfigFile} which are connected to an \class{Activity} instance via the \class{WasConfiguredBy} association class.
An \class{Activity} may thus be configured by a set of \class{Parameter} instances, by \class{ConfigFile} instances, or by a combination of both.

The corresponding description classes, \class{ParameterDescription} and \class{ConfigFileDescription}, are both defined in the context of the description of an activity.
There can be several instances of a \class{Parameter} (respectively \class{ConfigFile}) that are described by the same instance of \class{ParameterDescription} (respectively \class{ConfigFileDescription}).


\subsubsection{Parameter and ParameterDescription classes}
\label{sec:parameterandD}

\begin{table}[h!]
\small
\tymax  0.5\textwidth
 \textbf{\normalsize Parameter}\vspace{0.25em}\\
 \begin{tabulary}{1.0\textwidth}{llL}
 \toprule
 \head{Attribute} & \head{Data type}   & \head{Description}\\
 \midrule
\textbf{name}  & string & name of the parameter \\
\textbf{value} & string & the value of the parameter. If a corresponding \class{ParameterDescription}.\attribute{valueType} attribute is set, the value string can be interpreted by this \attribute{valueType}. \\
\bottomrule
\end{tabulary}
\caption[Attributes of the \class{Parameter} class]{Attributes of the \class{Parameter} class. Attributes in \textbf{bold} are mandatory and must not be null.}
\label{tab:param}
\end{table}

\begin{table}[h!]
\small
\tymax  0.5\textwidth
\textbf{\normalsize ParameterDescription}\vspace{0.25em}\\
\begin{tabulary}{1.0\textwidth}{lLL}
 \toprule
 \head{Attribute} & \head{Data type}   & \head{Description}\\
 \midrule
\textbf{name} & string & name of the parameter \\
\textbf{valueType} & VotableFieldFormat & description of a value from a combination of \attribute{datatype}, \attribute{arraysize} and \attribute{xtype} following VOTable 1.3 \citep[][, \S4.1]{2013ivoa.spec.0920O} \\
description & string  & a descriptive text for the parameter \\
unit        & Unit  & VO unit, see \ref{sect:Units} and \citet{2014ivoa.spec.0523D} for recommended unit representation \\
ucd         & string  & Unified Content Descriptor, supplying a standardized classification of the physical quantity, see \citet{2018ivoa.spec.0527M} \\
utype       & string  & Utype, meant to express the role of the parameter in the context of an external data model, see \citet{note:utypeusage} \\
min         & string & minimum value as a string whose value can be interpreted by the \attribute{valueType} attribute \\
max         & string & maximum value as a string whose value can be interpreted by the \attribute{valueType} attribute\\
options     & array of strings & array of possible values\\
default     & string & the default value of the parameter as a string whose value can be interpreted by the \attribute{valueType} attribute \\
\bottomrule
\end{tabulary}
\caption[Attributes of the \class{ParameterDescription} class]{Attributes of the  \class{ParameterDescription} class. Attributes in \textbf{bold} are mandatory and must not be null.}
\label{tab:Paramdescription}
\end{table}

The \class{Parameter} class contains a \attribute{value} and a \attribute{name} attribute that must be set (Table~\ref{tab:param}).

The \class{ParameterDescription} class describes the parameter \attribute{value} attribute similarly to the \class{ValueEntity} and \class{ValueDescription} classes. Those attributes are listed in Table~\ref{tab:Paramdescription}.

If a \class{ParameterDescription} instance is defined, the \attribute{name} attribute of the related \class{Parameter} instances must match the \attribute{name} attribute of this \class{ParameterDescription} instance.

The \class{Parameter} instance may refer to a \class{ValueEntity} instance using a \textit{hadReference} relation which gives the origin of the parameter value.


\subsubsection{ConfigFile and ConfigFileDescription classes}

\begin{table}[ht]
\small
\tymax  0.5\textwidth
 \textbf{\normalsize ConfigFile}\vspace{0.25em}\\
 \begin{tabulary}{1.0\textwidth}{llL}
 \toprule
 \head{Attribute} & \head{Data type}   & \head{Description}\\
 \midrule
\textbf{name} &  string & a human-readable name for the config file \\
\textbf{location} & string  &  a path to the config file, e.g., a URL/URI \\
comment & string  & text containing comments on the config file  \\
\bottomrule
\end{tabulary}
\caption[Attributes of the \class{ConfigFile} class]{Attributes of the \class{ConfigFile} class. Attributes in \textbf{bold} are mandatory and must not be null.}
\label{tab:configfile}
\end{table}

\begin{table}[ht]
\small
\tymax  0.5\textwidth
\textbf{\normalsize ConfigFileDescription}\vspace{0.25em}\\
\begin{tabulary}{1.0\textwidth}{llL}
 \toprule
 \head{Attribute} & \head{Data type}   & \head{Description}\\
 \midrule
\textbf{name}    & string & a human-readable name for the config file \\
\textbf{contentType}  & string  & format of the config file, MIME type when applicable \\
description     & string  & a descriptive text for the config file \\
\bottomrule
\end{tabulary}
\caption[Attributes of the \class{ConfigFileDescription} class]{Attributes of the  \class{ConfigFileDescription} class. Attributes in \textbf{bold} are mandatory and must not be null.}
\label{tab:configfiledescription}
\end{table}

The \class{ConfigFile} points to a structured, machine readable file, where parameters for running an activity are stored. It contains a \attribute{location} and a \attribute{name} that must be set, and a \attribute{comment} attribute (Table~\ref{tab:configfile}).

The \class{ConfigFileDescription} class indicates the format in which the content of the file is provided using a \attribute{contentType} attribute (see Table~\ref{tab:configfiledescription}).

If a \class{ConfigFileDescription} instance is defined, the \attribute{name} attribute of the related \class{ConfigFile} instances must match the \attribute{name} attribute of this \class{ConfigFileDescription} instance.


\subsubsection{Relations with Activity class}

\begin{table}[ht]
\small
\tymax  0.5\textwidth
 \textbf{\normalsize WasConfiguredBy}\vspace{0.25em}\\
 \begin{tabulary}{1.0\textwidth}{llL}
 \toprule
 \head{Attribute} & \head{Data type}   & \head{Description}\\
 \midrule
\textbf{artefactType} &  TypeOfConfigArtefact & literal that takes the value ``Parameter'' or ``ConfigFile'' to indicate the type of class pointed by the \class{WasConfiguredBy} instance. \\
\bottomrule
\end{tabulary}
\caption[Attributes of the \class{WasConfiguredBy} class]{Attributes of the \class{WasConfiguredBy} class. Attributes in \textbf{bold} are mandatory and must not be null.}
\label{tab:WasConfiguredBy}
\end{table}

The relation of \class{Parameter} and \class{ConfigFile} to \class{Activity} is formalized by a \class{WasConfiguredBy} class. There must be exactly one instance connected to a \class{WasConfiguredBy} instance, either a \class{Parameter} instance or a \class{ConfigFile} instance. The \class{WasConfiguredBy} class contains the attribute \attribute{artefactType} to indicate the type of class pointed by the \class{WasConfiguredBy} instance (see Table~\ref{tab:WasConfiguredBy}).

The life cycle of a \class{Parameter} instance (or \class{ConfigFile} instance) is the one of the corresponding \class{Activity} instance.
The life cycle of a \class{ParameterDescription} instance (or \class{ConfigFileDescription} instance) is the one of the corresponding \class{ActivityDescription} instance.
This means that when an activity is deleted from the provenance repository, its parameters and config files also disappear.

Several activities launched with various possible values for a parameter share the same \class{ParameterDescription} instance.
For instance, a cube analysis activity with a parameter ``nbofChannels'' will point to the corresponding instance of \class{ParameterDescription} (\attribute{name} = ``nbofChannels'', \attribute{ucd} = ``meta.number'', \attribute{unit} = Null, \attribute{description} = ``Nb of channel used for segmentation'').
`
Similarly, we can foresee a number of different \class{ConfigFile} instances used for various instances of an \class{Activity}, which rely on the same \class{ConfigFileDescription} instance bound to the corresponding \class{ActivityDescription} instance.



% make sure that images/tables of the previous section are printed
% before starting the new section
\clearpage
%\section{Applying provenance -- Interactions with other Data models}\label{sec:dmlinks}
\section{Links to other data models}
\label{sec:dmlinks}
%In this section we discuss how the Provenance Data Model interacts with
%classes and attributes from other VO data models (especially DatasetDM).
%(e.g. DatasetDM, SpectralDM (share some same classes), SimDM) 
%and how provenance information can be stored.

The Provenance Data Model can be applied without making any links to other 
IVOA data model classes. For example when the data is not yet published, provenance information
can be stored already, but a DatasetDM-description for the data may not yet exist.
However, if there are data models implemented for the datasets, then it is 
very useful to connect the classes and attributes of the other data models with Provenance classes and attributes (if applicable), which we are going to discuss in this Section. These links help to avoid 
unnecessary repetitions in the metadata of datasets, and also offer the possibility 
to derive some basic provenance information from existing data model classes automatically.


\subsection{Links with Dataset/Obscore Model}
Entities and their descriptions in the Provenance Data Model 
are tightly linked to the \class{DataSet}-class in the DatasetDM/ObsCore Data Model, as well as to 
InputDataset and OutputDataSet in the Simulation Data Model \citep[SimDM,][]{std:SimDM}.
Table \ref{tab:datasetmapping} maps classes and attributes from the Dataset Data Model 
to concepts in the Provenance Data Model.


%\begin{figure}[h]
%\centering
%\includegraphics[width=\textwidth]{../datamodel-diagrams/images/classes-relations-dms}
%\caption{Links between Agent and Party, Entity and Dataset.}
%\label{fig:class-relations-dm}
%\end{figure}
% --> a similar figure is already given in the sections on entity and agent.

\begin{table}[h]
\small
\tymax  0.5\textwidth
\begin{tabulary}{1.0\textwidth}{@{}Llp{4cm}@{}}
\toprule
\head{Provenance DM} & \head{Dataset DM} & \head{Comment}\\
\midrule
Entity.name                & DataID.title         & title of the dataset\\
HadMember.collectionId     & DataID.collection    & link to the collection to which the dataset belongs\\
Agent.name                 & DataID.creator       & name of agent\\
Entity.id                  & DataID.creatorDID    &  alternative id for the dataset given by the creator, could be used as Entity.id if no PublisherDID exists (yet)\\
WasGeneratedBy.activityId  & DataID.ObservationID & identifier to everything describing the observation\\
WasGeneratedBy.time        & DataID.date          & date and time when the dataset was completely created\\
Entity.id                & Curation.PublisherDID  & unique identifier for the dataset assigned by the publisher\\
Agent.id                 & Curation.PublisherID   & link to the publisher, i.e. to an Agent with role=``publisher''\\
Agent.name               & Curation.Publisher     & name of the publisher\\
Entity.releaseDate       & Curation.Date          & release date of the dataset\\
Entity.version           & Curation.Version       & version of the dataset\\
Entity.rights            & Curation.Rights        & access rights to the dataset; one of [...]\\
Entity.link              & Curation.Reference     & link to publication\\
Agent                    & Curation.Contact       & link to Agent with role contact\\
EntityDescription.dataproduct\_type & DataProductType  & the type of a dataproduct from Dataset Metadata Model can be used as attribute to entity\\
EntityDescription.dataproduct\_subtype & DataProductSubType & subtype of a \mbox{dataproduct}/entity\\
EntityDescription.level & ObsDataset.calibLevel  & (output) calibration level, integer between 0 and 3\\\hline
\bottomrule
\end{tabulary}
\caption{Mapping between attributes from Dataset Metadata Model classes to classes in ProvenanceDM}
\label{tab:datasetmapping}
\end{table}


\begin{figure}[h]
\centering
\includegraphics[width=\textwidth]{../datamodel-diagrams/images/agent-relations.pdf}
\caption{The relations between the \class{Agent} class within the Provenance Data Model 
(grey and yellow classes) with classes from the Dataset Metadata Model, party package (green).}
\label{fig:agent-relations}
\end{figure}

The \class{Agent} class, which is used for defining responsible persons and 
organizations in ProvenanceDM, is very similar to the \class{Party} class in the Dataset Metadata Model (and in SimDM). Its details are depicted in Figure~\ref{fig:agent-relations}.
The main difference between \class{Agent} and \class{Party} is that \class{Individual} and \class{Person} are subclasses in DatasetDM, whereas we just use the same class \emph{Agent} for both and distinguish between them using the \emph{Agent.type} attribute (which can have the value ``Individual'' or ``Organization'').


We imagine that services implementing both data models, \class{Dataset} and \class{ProvenanceDM} may use just \emph{one} class: either \class{Agent} or \class{Party}, enriched with all the necessary (project-specific) attributes. Note that for Provenance queries using a ProvTAP service and for W3C compatible serializations, the name \class{Agent} for the responsible individuals/organizations is required.



\subsection{Links with Simulation Data Model}
In SimDM one also encounters a normalization similar to our split-up of descriptions from 
actual data instances and executions of processes: the SimDM class ``experiment'' 
is a type of \class{Activity} and its general, reusable description is called a ``protocol'',
which can be considered as a type of this model's \class{ActivityDescription}. 
More direct mappings between classes and attributes of both models are given in Table~\ref{tab:simdmmapping}.

\begin{table}[h]
\small
\tymax  0.5\textwidth
\begin{tabulary}{1.0\textwidth}{@{}Llp{4cm}@{}}
\toprule
\head{Provenance DM} & \head{Simulation DM} & \head{Comment}\\
\midrule
Activity               & Experiment      &  \\
Activity.name         & Experiment.name & human readable name; name attribute in SimDM is inherited from Resource-class\\
Activity.endTime & Experiment.executionTime  & end time of the execution of an experiment/activity \\
Activity.activityDescription & Experiment.protocol & reference to the protocol or description class \\
ActivityDescription    & Protocol        & \\
ActivityDescription.name  & Protocol.name   & human readable name\\
ActivityDescription.doculink & Protocol.referenceURL & reference to a webpage describing it\\
% add Protocol.code, Protocol.version?
Parameter              & ParameterSetting     & value of an (input) parameter\\
ParameterDescription   & InputParameter       & description of an (input) parameter\\
Agent           & Party           & responsible person or organization\\
Agent.name      & Party.name      & name of the agent \\
WasAssociatedWith & Contact         & \\
WasAssociatedWith.role & Contact.role    & role which the agent/party had for a certain experiment (activity); SimDM roles contain: \texttt{owner}, \texttt{creator}, \texttt{publisher}, \texttt{contributor}\\
WasAssociatedWith.agent & Contact.party    & reference to the agent/party \\
Entity        & DataObject     & a dataset, which can be/refer to a collection\\

\bottomrule
\end{tabulary}
\caption{Mapping between classes and attributes from ProvenanceDM to classes/attributes in SimDM.}
\label{tab:simdmmapping}
\end{table}



% Remove this, because no further links to other data models are currently planned.
%\subsection{Further links to data models}
%More similarities and links to other data models will be detailed in future
%versions of this working draft.



% make sure that images/tables of the previous section are printed
% before starting the new section
\clearpage
\section{Accessing provenance information}
\subsection{Access protocols}
\label{sec:access_protocols}
We envision two possible access protocols:
\begin{itemize}
\item ProvDAL: retrieve provenance information based on given ID of a data entity or activity.
\item ProvTAP: allows detailed queries for provenance information, discovery of datasets based on e.g. code version.
\end{itemize}

\subsection{ProvDAL}
\subsubsection{ProvDAL}
ProvDAL is a service the interface of which is organized around one main parameter, the \urlparam{\bf ID} of an entity (obs\_publisher\_did of an ObsDataSet for example), activity or an agent.
The response is given in one of the following formats: \urlparam{PROV-N}, \urlparam{PROV-JSON}, \urlparam{PROV-XML}, \urlparam{PROV-VOTABLE}.
Additional parameters can complete the \urlparam{ID} to refine the query: \urlparam{\bf FORMAT} allows to choose the output format. \urlparam{\bf DEPTH} gives the number of relations that shall be tracked along the provenance history, independent of the type of relation. Its value is either 0, a positive integer or \urlparam{ALL}. If this parameter is omitted, the default is \urlparam{ALL}, which returns the complete provenance history that the service has stored or the provenance according to a maximum depth number that the server allows.

The \urlparam{ID} parameter is allowed more than once in order to retrieve provenance details for several activities or datasets at the same time. Here are a few example requests:

\begin{verbatim}
{provdal-base-url}?ID=rave:dr4&FORMAT=PROV-JSON
{provdal-base-url}?ID=rave:dr4&ID=rave:act_irafReduction&DEPTH=2
\end{verbatim}

\noindent
The format can also be specified via the HTTP accept header, e.g.
\begin{verbatim}
wget -d --header="Accept: application/json" \
   {provdal-base-url}?ID=rave:dr4
\end{verbatim}
would return the provenance information in \urlparam{PROV-JSON} format.
\noindent
If both \urlparam{FORMAT} and the accept header are used and \urlparam{FORMAT} specifies a format that is incompatible with the HTTP accept header, then the service should return with a HTTP status 406: Not Acceptable.

For services which allow tracking the provenance information forward, e.g. in order to check for which activities an entity was used, the optional parameter \urlparam{\bf DIRECTION} can be set to \urlparam{FORTH}. Its default value is \urlparam{BACK}. This influences the direction in which the used, wasGeneratedBy, wasDerivedFrom and wasInfluencedBy relations are followed.

The provenance data model defines also the hierarchical relations \emph{hadMember} for entity collections and \emph{hadStep} for activityFlows. If a node belongs to a collection or activityFlow, these relations shall be returned as well, independent of the specified tracking direction.
If one is interested in more details and wants to follow the \emph{members} of an entity collection or the \emph{steps} of an activityFlow, these can be included by setting the optional parameter \urlparam{\bf MEMBERS} or \urlparam{\bf STEPS} to \urlparam{TRUE}, respectively. The default is \urlparam{FALSE}.

By default, it is recommended to stop any further tracking at an agent node, unless an additional optional parameter \urlparam{\bf AGENT} is set to \urlparam{TRUE}. Note that this means that the request for any agent will always return just the agent node itself and nothing else, unless \urlparam{AGENT=TRUE} is used. Thus, if one wants to know which entities and activities an agent has influenced, the request looks like this:

\begin{verbatim}
{provdal-base-url}?ID=org:rave&AGENT=TRUE&DEPTH=1
\end{verbatim}

\noindent
\urlparam{DEPTH=1} was used here in order to avoid following the found entities and activities any further.

%\comment{Maybe it's better to use DEPTH and DIRECTION instead of FORWARD and BACKWARD. Reason: if a service just implements the backward direction, then it's weird to call something ``backward'' if there is no ``forward'' as well. DEPTH is also a commonly used word when refering to graphs and numbers of relations.}


\begin{figure}[h]
\centering
\includegraphics[width=1.0\textwidth]{provenance-graph-example-depth2.pdf}
\caption{An example provenance graph, highlighting the objects and relations returned from a ProvDAL service with ID=E6 and \urlparam{DEPTH}=2. The \urlparam{BACK} and \urlparam{FORTH} values for \urlparam{DIRECTION} are only important for the processing relations (solid lines). Hierarchial (dashed) and responsibility relations (dotted) are only followed ``upwards'' and towards agents by default. If they should also be followed in the other direction, then the additional optional parameters \urlparam{MEMBERS}, \urlparam{STEPS} and \urlparam{AGENT} need to be set to \urlparam{TRUE}.}
\label{fig:provenance-graph-example}
\end{figure}


A ProvDAL service MUST implement the parameters \urlparam{ID}, \urlparam{DEPTH} and \urlparam{FORMAT}; the remaining parameters are optional.
If a service does not implement the optional parameters, but they appear in the request, then the service should return with an error.

Table~\ref{tab:provdal-parameters} summarizes the parameters for such a ProvDAL service interface.

\begin{table}[h]
\small
\begin{tabulary}{1.0\textwidth}{@{}p{0.17\textwidth}p{0.22\textwidth}p{0.53\textwidth}@{}}
%{llp{0.2\textwidth}p{0.3\textwidth}}
\toprule
\head{Parameter} & \head{Value/options} & \head{Description}\\\hline
\midrule
\textbf{\urlparam{ID}} & qualified \urlparam{ID} & a valid qualified identifier for an entity or activity (can occur multiple times)\\
\textbf{\urlparam{DEPTH}} & 0,1,2,..., \urlparam{\underline{ALL}} &  number of relations to be followed or \texttt{ALL} for everything, independent of the relation type\\
\textbf{\urlparam{FORMAT}} & \urlparam{PROV-N}, \newline\urlparam{PROV-JSON}, \newline\urlparam{PROV-XML}, \newline\urlparam{PROV-VOTABLE} & serialisation format of the response\\\hline
\urlparam{DIRECTION} & \urlparam{\underline{BACK}}, \urlparam{FORTH} & \urlparam{BACK} = track the provenance history, \newline\urlparam{FORTH} = explore the results of activities and where entities have been used\\
\urlparam{MEMBERS} & \urlparam{TRUE} or \urlparam{\underline{FALSE}} & if \urlparam{TRUE}, retrieve and track members of collections\\
\urlparam{STEPS} & \urlparam{TRUE} or \urlparam{\underline{FALSE}} & if \urlparam{TRUE}, retrieve and track steps of activityFlows\\
\urlparam{AGENT} & \urlparam{TRUE} or \urlparam{\underline{FALSE}} & if \urlparam{TRUE}, retrieve all relations for agents, i.e. find out what an agent is responsible for\\
\bottomrule
\end{tabulary}
\caption{ProvDAL request parameters. Options that are \textbf{required} to be implemented by ProvDAL services are marked with bold face. \underline{Default} values are underlined.}
\label{tab:provdal-parameters}
\end{table}





\subsection{ProvTAP}
% update from Fran�ois / 2017 15/09
ProvTAP is a TAP service specialisation for delivering IVOA Provenance metadata. As any TAP service (reference to TAP) it is providing ADQL ``query responses'' made of a single table gathering columns selected from the various tables defined in the service TAP schema. By default this table is formatted as a VOTABLE. Other formats (json, tsv, etc...) could be added optionnaly.

The table and column definitions with their metadata (unit, ucd, utype, datatype, description) available in the TAP schema are exactly the ones defined in the PROV-VOTABLE mapping of the IVOA provenance data model (see table and column definition  and metadata in the Appendix). This is totally similar to what has been done in the RegTAP specification (reference) for mapping the VOresource datamodel.

Example of ADQL queries for ProvTAP are also given in the appendix
%end of update 

\TODO{We need more details here! Output of TAP service is NOT a PROV-VOTABLE by default!}

%\TODO{Do we need combined query possibilities, i.e. ask for ObsCore-fields and Provenance fields in one query? Or rather use a 2-step-process, decoupling them from each other?}


%\TODO{Also look at PROV-AQ from the W3C.}

\subsection{VOSI availability and capabilities}
According to the DALI specification for VO services \citep{std:DALI}, a provenance service implementing ProvDAL and/or ProvTAP must provide a VOSI availability interface as well as a capabilities interface with entries for ProvDAL and/or ProvTAP. The \texttt{standardId}s for these provenance interfaces are:

\begin{verbatim}
ivo://ivoa.net/std/ProvenanceDM#ProvDAL-1.0
ivo://ivoa.net/std/ProvenanceDM#ProvTAP-1.0
\end{verbatim}

The capability for a TAP service to support the Provenance DM is expressed by the dataModel element as :
\begin{verbatim}
<dataModel ivoid="ivo://ivoa.net/std/ProvenanceDM#core-1.0">ProvenanceDM-1.0</dataModel>
\end{verbatim}

For ProvTAP, the VOSI tables interface also needs to be provided.




\section{Discussion}
\subsection{Links, ids}\label{sec:links_between_data}
It would be convenient, if each data object or even each file 
gets a unique id that can be referenced. The W3C provenance model requires ids
for entities, activities and agents, and they have to be qualified strings, 
i.e. containing a namespace. For example, an activity in the RAVE-pipeline could 
have the id `\texttt{rave:radialvelocity\_pipeline\_20160901}'. Using a namespace for each 
project for these ids will help to make them unique. 

If several copies of a dataset exist, and one of them is corrupted, it would even be useful to know
exactly which copy was used by a given activity. This can be modeled already 
with the existing tools (using a copy-activity), but we doubt that many people
would actually need this level of detail.

IVOIDs and DOI's are potentially good candidates for unique identifiers.


%\subsubsection{Calibration data}
%The calibration dataset consists of images that can be used to calibrate the
%raw data. It is not necessary to mention them explicitly in the model, 
%they are just another dataset that is used by activities with a 
%calibration-method.

%\subsubsection{Quality}
%For expressing the quality of data, we could simply define additional 
%attributes for each \class{Activity}
%or \class{Entity} object, i.e. zero, one, or more properties in the form of
%key-value pairs. We could use a \class{Quality} namespace to mark a keyword
%as quality-related:
%\begin{itemize}
%    \item quality:comment: [some text]
%    \item quality:seeing: [some value]
%\end{itemize}
%The values could range from a float number to free text.


%\subsubsection{Provenance of provenance}
%``Bundles'' are used to name a set of provenance descriptions. It is a type for 
%an entity, and allows to express provenance of provenance. This is probably  
%very interesting for workflow systems.
% -- partially covered already with ActivityFlow

\subsection{Description classes}
This model was established mainly having a database implementation in mind. 
However, it may be better in the long run to store provenance with 
the entities themselves, e.g. as an additional extension in fits-headers.

A model using description classes for defining templates for activities and
entities has an advantage for normalization: the common processes could be 
described once and for all at some place and then be reused when recording
provenance information for certain entities and activities. This \emph{some place} 
is actually the crucial point here.
In an ideal world, ``some place'' could collect all the descriptions from all 
the possible datasets and methods in astronomy, but building such a look-up place 
is a quite challenging task -- it will probably never be complete. There's also 
the issue of persistent identifiers/broken links to consider.
Normalisation is useful for closed systems, e.g. for describing the provenance 
for data produced by a certain pipeline (e.g. MuseWise system) or with 
workflow tools or when a task needs to be repeated many times. However, the VO 
is quite the contrary of a closed system and we need to keep an eye on what is 
actually achievable.

When writing down a simple serialisation of e.g. the provenance for a stacked 
image using the current model including the description classes, it soon becomes quite cumbersome to define 
everything twice: first the descriptions, then the instances. This basically 
doubles the number of entries to describe provenance (unless there is already 
some place with all the descriptions to which we can refer).

Expressing provenance for a stacked image with this smaller set of classes may 
be simpler, but on the other hand constructing a database schema becomes much 
harder. 
We could leave it to the implementors to choose what is more useful for them.
When extracting a serialisation of the provenance information from a provenance 
service, the attributes of the description classes could be combined with 
the corresponding activity/entity classes. This will produce some repetition
(e.g. many entities may have the same descriptive attributes), but 
avoid having too many classes and links between them.
% Note: Harry Enke commented that this sentence is not understandable; 
% we can remove this sentence later on when we have a proper implementation-note section.
%\Note{Descriptions could be present in W3C-conform serialisations, if we 
%put them into entities.}

%\TODO{Check, if PROV-Templates from the W3C (inofficial note) could be used 
%for ActivityDescriptions.}

\subsection{ActivityFlow and implications for multiplicities}
By introducing the \class{ActivityFlow} class, one entity can now have many 
wasGeneratedBy-links to activities. One of them would be the actual generation-activity, 
the other activities can only be activityflows containing this generation-activity.
This is not expressed explicitly in the current model. 

We could introduce an additional abstract class, e.g. \class{AbstractActivity}, with \class{Activity} and 
\class{ActivityFlow} being subclasses to this one. But this adds another layer of complexity 
that we may not want in this data model.

Since we introduced \class{ActivityFlow} mainly for having different view levels, 
we may want to add an attribute \emph{viewLevel} to descriptions of activityflows.

We are planning to test how it all works in implementations, which classes and attributes are 
needed or not and will then adjust the model 
accordingly.

\subsection{VO-DML representation}
We do not yet have a VO-DML compliant representation of the model. This is one 
of the issues to be clarified for the next version.

\subsection{Links to other data models}
Section~\ref{sec:dmlinks} still needs to be expanded further, especially making detailed links with the 
Simulation Data Model will be very useful.



\section{Implementations of the data model for specific use cases}
\label{sec:usecases-implementations}
%\subsection{One processing step in PROV-N notation}
%
%\TODO{Put the very simple example here}
%See \url{https://volute.g-vo.org/svn/trunk/projects/dm/provenance/description/prov-example-incl-prototypes.txt}
%and \url{https://volute.g-vo.org/svn/trunk/projects/dm/provenance/description/prov-example-w3c.txt}

This section presents some general guidelines for applying the data model and
%This section presents some 
specific use cases for which the provenance data model helps to solve certain tasks. Details on specific implementions of the provenance data model are provided in a separate document, the ProvenanceDM Implementation Note \citep{std:ProvenanceImplementationNote}.


\subsection{How to use the data model}
%\TODO{KR: I think this is a good place for this section. Do we still want to have this section or is everything covered with the new section on entitydescription-serialisation?}
%\begin{itemize}
%\item identify entities in your project, i.e. the things you deal with
%\item identify activities (processes) in your project
%\item identify reponsible agents (persons and organizations)
%\item find the relations between them
%\item possibly disentangle entity properties and entityDescription properties: everything that you may know about an entity before its creation, belongs to the entityDescription; e.g. file format, dataProduct\_type, what kind of entity it is going to be (category)
%\item ...
%\end{itemize}

The IVOA Provenance data model has been developed along with its implementations from different projects. 
We gather here some tips to implement and use the model for a specific project.

\paragraph{Before using the model}
We noticed that the simple knowledge of what is provenance information is important for the conception of all projects. Before using or not the Provenance data model and associated services, it is good practice to locate and collect information on the activities, entities and agents that will be manipulated, and be sure that this information is not lost along the way. For example, a script may use intermediate files (such as calibration files for observations) that may not be tracked by the system.

\paragraph{Define unique identifiers}
It would be convenient if each data object or even each file 
gets a unique id that can be referenced. The W3C provenance model requires ids
for entities, activities and agents, and they have to be qualified strings, 
i.e. containing a namespace. For example, an activity in the RAVE-pipeline could 
have the id `\texttt{rave:radialvelocity\_pipeline\_20160901}'. 
Using a namespace for each 
project for these ids will help to make them unique. 
IVOIDs, DOI's or ORCIDs are potentially good candidates for unique identifiers.

\paragraph{Use of the description classes} 
One may only use the core data model without the description classes if they are not needed for the project. In that case, it is recommended to merge the attributes of the description classes into the main classes (Entity/Activity/Agent, and if needed Parameter). In the same way, when serializing the provenance information, the description classes can be merged to the main classes, which is needed to produce W3C compliant provenance files.

\paragraph{Add project specific attributes to entities, activities and agents}
We proposed generic attributes for the different classes, but there are probably project specific attributes that need to be added. 
It is also required at this level to disentangle Entity properties and EntityDescription properties: everything that you may know about an entity before its creation, belongs to the EntityDescription (e.g. file format, content\_type, dataproduct\_type, category, \dots).

\paragraph{Identify entities, activities and agents}
If inside the project, the different entities that will be manipulated are already defined precisely, as well as the activities producing them, then the description classes are probably of interest and will help reduce the redundancy in the provenance information stored. 

\paragraph{Create ActivityDescription files}
A model using description classes to define templates for activities and
entities has an advantage for normalization: the common processes could be
described once and for all at some place and then be reused when recording
provenance information for certain entities and activities.
If description classes are relevant for a projet, it may be sufficient to store the descriptions directly as ActivityDescription files (see Section \ref{sec:description-serialization}). 
Those can be created and centralized before implementing a provenance database or service.

\paragraph{Link to an Authentication System}
We proposed to add optional attributes to tha Agent class (email, phone, address). Some projects may include an authentication system with a user directory. In that case a link should be kept between the Agent identifiers and the ones used in the authentication system when it is relevant, and information should be synchronized. However, there may be some agents that are not defined in the authentication system, so the information may not be easy to merge.

\paragraph{Adding additional metadata to an existing Entity}
It could happen that after creating entities and storing their provenance, additional metadata is needed for those entities. For example, an ObsCore description may have to be added to some entities when they are made available to the public. In that case, the entity will receive an external identifier (e.g. a publisher\_did), which should thus be associated to the entity provenance identifier (Entity.id) in a relation table.


\subsection{voprov Python package}\label{sec:implementation_voprov}
The voprov package is an open source python library derived from the prov python library (MIT license) developed by Trung Dong Huynh (University of Southampton). 

The prov package implements the W3C Provenance Data Model.  It offers to describe the provenance in different format: PROV-N, JSON, XML, etc. and provides different output formats of the serialized data: PROV-N, JSON, and graphic formats: PDF, PNG, SVG.

The voprov library allows users to describe the provenance of their data according to the IVOA Provenance Data Model. It allows the description of flows of activities (pipelines) with eventually the composition of the different steps. It provides the VOTABLE serialization. 

This library is currently used in the context of the Pollux database which hosts synthetic stellar spectra. The provenance files are created from the non normalized information found in the Pollux header files. The serialization is proposed in 3 detail levels and in different formats: on one one hand in PROV-N, JSON, XML and VOTABLE formats and on the other hand in PDF, PNG and SVG graphic formats. The VO user or the VO tool is informed of the existence of the provenance in a DATALINK section of the SSA response which gives information on how to retrieve a given provenance file.

%This code writes the serialisation examples which are given in Section~\ref{sec:serialisations}.

Example code and serializations are given in the Implementation Note \citep{std:ProvenanceImplementationNote}.


\subsubsection{Graphic formats}
\label{sec:graphic_formats}
The voprov python module can also provide provenance information in graphic formats: PNG, SVG and PDF.
In the above example, you have to add the following instructions in your python program:

\begin{verbatim}
    dot = prov_to_dot(provdoc, use_labels=True)
    dot.write_png('ex1.png')
    dot.write_svg('ex1.svg')
    dot.write_pdf('ex1.pdf')
\end{verbatim}

\begin{figure}
\centering
\includegraphics[width=0.9\textwidth]{access_ex1.png}
\caption{Example: png format@}
\label{fig:example}
\end{figure}



\subsection{Provenance of RAVE database tables (DR4)}
The RAVE survey (Radial Velocity Experiment) recorded spectra for about half a 
million stars. These spectra are processed in a number of steps until the 
derived properties are published in the RAVE data releases at http://www.rave-survey.org.
Providing provenance information for the data, from which spectrum and fibre the
data was coming from and which steps were involved in processing the data, can help scientists
to understand the data and their restrictions and judge their quality.
It would also be useful to be able to compare if, how and why the derived data 
for some stars have changed between different releases.
Provenance information for some major steps of RAVE DR4 was loaded in 
W3C-compatible PROV-N notation and uploaded to the provenance store at 
https://provenance.ecs.soton.ac.uk/store/documents/84064/. This allows to view 
graphs of the workflow by visualising only the main entities, activities and agents 
with their relations. It shows that the provenance concepts explained in this draft 
can be applied directly to data obtained from astronomical observations.

We also tested a Django implementation of the classes in this document along with provenance data stored in an SQLite database. This allows to quickly setup a provenance web service
which gives the possibility to view all instances of a class or details for a single object, 
extract provenance information for single entities (backwards in time) and 
visualise the provenance information. 
More details about this are available in the implementation notes \citep{std:ProvenanceImplementationNote}.




\subsection{Provenance for CTA}

The Cherenkov Telescope Array (CTA) is the next generation ground-based very high energy gamma-ray instrument. It will provide a deep insight into the non-thermal high-energy universe. Contrary to previous Cherenkov experiments, it will serve as an open observatory providing data to a wide astrophysics community, with the requirement to propose self-described data products to users that may be unaware of the Cherenkov astronomy specificities. The proposed structure of the metadata is presented in Figure~\ref{fig:cta_dm}.

\begin{figure}
\centering
\includegraphics[width=\textwidth]{CTA_DM_high_level.png}
\caption{CTA high level data model structure with Pipeline stages and connection to IVOA ProvenanceDM.}
\label{fig:cta_dm}
\end{figure}

Cherenkov telescopes indirectly detect gamma-rays by observing the flashes of Cherenkov light emitted by particle cascades initiated when the gamma-rays interact with nuclei in the atmosphere. The main difficulty  is that charged cosmic rays also produce such cascades in the atmosphere, which represent an enormous background compared to genuine gamma-ray-induced cascades. Monte Carlo simulations of the shower development and Cherenkov light emission and detection, corresponding to many different observing conditions, are used to model the response of the detectors.  With an array of such detectors the shower is observed  from several points and, working backwards, one can figure out the origin, energy and time of the incident particle. The main stages of the CTA Pipeline are presented inside Figure~\ref{fig:cta_dm}. Because of this complexity in the detection process, provenance information of data products is necessary to the user to perform a correct scientific analysis.

Provenance concepts are relevant for different aspects of CTA :
\begin{itemize}
\item Data diffusion: the diffused data products have to contain all the relevant context information with the assumptions made as well as a description of the methods and algorithms used during the data processing.
\item Pipeline: the CTA Observatory must ensure that data processing is traceable and reproducible.
\item Instrument Configuration: the characteristics of the instrument at a given time have to be available and traceable (hardware changes, measurements of e.g. a reflectivity curve of a mirror, ...)
\end{itemize}

We tested the tracking of Provenance information during the data analysis using the Python prov package inside OPUS\footnote{\url{https://github.com/ParisAstronomicalDataCentre/OPUS}} (Observatoire de Paris UWS System), a job control system developed at PADC (Paris Astronomical Data Centre). This system has been used to run CTA analysis tools and provides a description of the Provenance in the PROV-XML or PROV-JSON serialisations, as well as a graph visualization (see Figure~\ref{fig:cta_prov}).

\begin{figure}
\centering
\includegraphics[width=0.8\textwidth]{CTA_prov.png}
\caption{Provenance description of a CTA analysis step.}
\label{fig:cta_prov}
\end{figure}

The CTA Pipeline contains a specific Provenance class dedicated to the collection of provenance information after each processing step. This informations is returned as an output file for now.

More details about the related implementations are available in the implementation notes \citep{std:ProvenanceImplementationNote}.


\subsection{Provenance for the POLLUX database}

POLLUX is a stellar spectra database proposing access to high resolution synthetic spectra computed using the best available models of atmosphere (CMFGEN, ATLAS and MARCS), performant spectral synthesis codes (CMF\_FLUX,SYNSPEC and TURBOSPECTRUM) and atomic linelists from VALD database and specific molecular linelists for cool stars. 

Currently the provenance information is given to the astronomer in the header of the spectra files (depending on the format: FITS, ascii, xml, votables, ...) but in a non normalized description format. 

The implementation of the provenance concepts in a standardized format allows users on one hand to benefit from tools to create, visualize and transform in another format the description of the provenance of these spectra and on a second hand to select data depending on provenance criteria.

\begin{figure}
\centering
\includegraphics[width=0.9\textwidth]{usecase_Pollux_example1.png}
\caption{Pollux Example 1}
\label{fig:pollux}
\end{figure}

\subsection{Provenance of HiPS datasets}
HiPS is a new all sky organization of pixel data. It is based on HealPix tesselation of the sky on equal area cells (pixels) for a given HealPix order gathered in tiles. Adaptative resolution is achieved by a hierarchy of tiles at increasing order. Sorting and organization is based on a tree of including directories each of those associated with a tile. HiPS specification has entered the IVOA recommendation process and is becoming an interoperability standard.
In the processing chain, HiPS can be seen as a kind of ``legacy level'' for observational data.


An HiPS dataset can be generated either by Aladin in ``hipsgen'' mode or by other softwares.
%\cite{if others}.  
The processing distinguishes 3 main different methods for estimating cell values with parameters: FIRST(nearest neighbour), MEAN and MEDIAN of the neighboring pixels. Up to 50 parameters can help to tune the processing, among which the higher resolution HealPix order, the sky background value to be substracted, the border width or the mask to apply to original images to avoid including bad area in the computing, etc.


An example of provenance metadata for a HiPS collection generated from a collection of SERC Schmidt plates scanned by CAI-Observatoire de Paris with the MAMA facility and serialized in PROV-N format is given at \\\footnotesize
\footnotesize{\url{https://volute.g-vo.org/svn/trunk/projects/dm/provenance/example/HiPS-prov-provn.txt}}, the corresponding VOTable-format is available at \\
\url{https://volute.g-vo.org/svn/trunk/projects/dm/provenance/example/HiPS-prov-vot.xml}.

Here is an excerpt of the corresponding PROV-N serialization:

\begin{verbnobox}[\scriptsize]
prefix ivo <http://www.ivoa.net/documents/rer/ivo/>
prefix ex <http://www.example.com/provenance/>
prefix voprov <http://www.ivoa.net/documents/dm/provdm/voprov/>

Entity
( ivo://CDS/P/MAMA/ESO-R, 
[ voprov:name = "ESO-R MAMA HIPS at CDS",
voprov:annotation = "HiPS version of ESO Schmidt survey digitized by Mama and processed by CDS",
voprov:type= "voprov:entity",
voprov:access_reference = "http://CDS/P/MAMA/ESO-R", // as defined in obscore 
voprov:doculink = "http://cds.u-strasbg.fr/hips/documentation.html#structure",
voprov:level = 3,
hips:dataproduct_type = "voprov:hips_pixels",
hips:HiPS_properties = "http://cds.u-strasbg.fr/hips/p/mama/eso-r/properties.txt"] )

// Relationship
WasAttributedTo(ivo://CDS/P/MAMA/ESO-R, ivo://cds, prov:role= "voprov:creator")

Agent
(ivo://cds,
[ voprov:name= "CDS",
voprov:email = "question@astro.unistra.fr",
voprov:type = "Organisation" ]) 

WasGeneratedBy  (ivo://CDS/P/MAMA/ESO-R, EHG1, -) 

Activity
(EHG1,
[ voprov:name = "ESO HiPS generation 1",
voprov:startTime = "2016-07-18", 
voprov:endTime = "2016-07-20",
voprov:annotation = "Final generation activity of HiPS for ESO Mama survey",
voprov:activityDescription = "HipsgenM"] )
 
ActivityDescription 
(HipsgenM,
[ voprov:name = "HiPSgen_Mean",
voprov:type = "data encoding",
voprov:subtype= "HiPSgen",
voprov:doculink = "http://cds.u-strasbg.fr/HiPSGEN-Documentation"])

WasAssociatedWith( EHG1, Buga, voprov:role="voprov:operator")
WasAssociatedWith(EHG1, ivo://CDS, prov:role="voprov:creator")
WasAttributedTo(( ivo://CDS/P/MAMA/ESO-R, buga, prov:role= "voprov:operator")
\end{verbnobox}



\appendix
\section{Changes from Previous Versions}
% No previous versions yet.
% these would be subsections "Changes from v. WD-..."
% Use itemize environments.
\subsection{Changes from WD-ProvenanceDM-1.0-20161121}
\begin{itemize}
\item Use parameters \texttt{BACKWARD} and \texttt{FORWARD} instead of \texttt{STEP} in ProvDAL, Section~\ref{sec:access_protocols}.
\item Expanded Section~\ref{sec:access_protocols} on access protocols, especially ProvDAL part.
\item Moved the figure showing relations between Provenance.Agent and Dataset.Party into Section~\ref{sec:dmlinks}.
\item Added serialisation examples to Section~\ref{sec:serialisations}, voprov-implementation example to Section~\ref{sec:usecases-implementations}.
\item Use voprov:type and voprov:role in Table~\ref{tab:agent-roles} with example agent roles, i.e. replaced prov:person by Individual and prov:organization by Organization.
\item Removed the obscore/dataset attributes from EntityDescription, since they are specific for observations only and are not applicable to configuration entities etc.
\item Renamed \emph{label} attribute to \emph{name} everywhere, for more consistency with SimDM naming scheme (\emph{label} is reserved there for SKOS labels).
\item Extended the entity role examples in table \ref{tab:entity-roles}.
\item Renamed attribute \emph{Entity.access} to \emph{Entity.rights} for more consistency with DatasetDM etc.
\item Moved detailed implementation section from appendix to a separate document (implementation note), shortened the use cases \& implementation section.
\item Removed description\_ref as attribute, since it's expressed by the corresponding link in the model anyway.
\item More explanations on links to data models in Section~\ref{sec:dmlinks}.
\item Introduced subsections for Section~\ref{sec:dmlinks}, added table with SimDM-links.
\item Renamed \emph{docuLink} to \emph{doculink}
\item Avoid double-meaning of \emph{description} by splitting it up into: 
    \begin{itemize}
    \item \emph{description\_ref}: a foreign key, reference to a description class 
(which could be located at an url as well)
    \item \emph{annotation}: free text description
    \end{itemize}
\item Applied similar naming scheme to \emph{Parameter} and \emph{ParameterDescription}-classes
\item Renamed Section~\ref{sec:usecases-implementations} to stress that it deals with implementations.
\item Added links to provn and votable-serialization for HiPS-use case, added first part of provn as example in the HiPS-use case section.
\item Corrected attribute names in Table~\ref{tab:datasetmapping}.

\end{itemize}


% \section{Implementation details}\label{sec:implementation-details}
% In this section we will give more details on the classes and attributes which were used 
% in implementations for each use case. This maybe needs to go into a different document, so it can 
% be updated without affecting this standard.

% TBD.


\bibliography{ivoatex/ivoabib,prov-refs}


\end{document}
