
Serialization files constitute the building blocks of the client/server dialogs.
The provenance information as represented in the data model is split in three main concepts that can be searched following many different relations between the main 3 classes, \class{Activity}, \class{Entity} and \class{Agent}.
The selection of the relations to expose when distributing the provenance information depends on the usage and will be described more extensively in the Implementation Note \citep{std:ProvenanceImplementationNote} and the links therein.

To give a very simple example, suppose a client asks for the context of execution for one specified activity, which computes a simple RGB color composition. 
On the server side, exposing the provenance information for this activity or for an entity, corresponding to a monocolor or RGB image, 
means expose only the structure of the classes
and relation tables and feed them with the related tuples in the database.
On the client side, the content of a VO-Provenance serialization document can then be explored and represented using graphical interfaces, as inspired by the Provenance Southampton suite or by customized visualisation tools.

\subsection{W3C serialization formats: PROV-N, PROV-JSON and PROV-XML}
% from W3C reused and extended

Serialization formats are proposed in the W3C PROV framework for storing and exchanging the provenance metadata: {PROV-N}, {PROV-JSON},{PROV-XML} and PROV-RDF, that are defined in \cite{std:PROV-N}, \cite{std:PROV-JSON}, \cite{std:PROV-XML},and \cite{std:PROV-O} respectively. 
%These are serializations of the W3C provenance data model, a larger set of classes and relations compared to this model but sharing the same core structure. 
They can be reused here as well for serializations of our data model. 

In order to produce W3C compatible serializations, the classes and attributes defined in the IVOA Provenance DM model must be qualified names with the namespace \texttt{voprov}, except when they exist in the W3C PROV namespace \texttt{prov} (e.g. \texttt{prov:id}, \texttt{prov:type} and \texttt{prov:startTime}). A mapping is given in e.g. Tables~\ref{tab:entity}, \ref{tab:wasderivedfrom}, \ref{tab:activity}, \ref{tab:wasinformedby}, \ref{tab:used} and \ref{tab:wasgeneratedby}. 

The specialized entities defined in Section~\ref{sec:spec_entities} should be written as Entity instances and the attribute \attribute{prov:type} must be set to the category of the specialized entity e.g. \attribute{voprov:Data}, \attribute{voprov:Parameter}, \attribute{voprov:ActivityDescription}. This is also the rule for \class{Collection}, as done in W3C PROV-DM. We note here that several \attribute{prov:type} can be provided.

The specialized relations defined in Section~\ref{sec:spec_relations} should be written as a W3C relation (e.g. used relation between an Activity and and Entity, or the more general wasInfluencedBy relation between all classes). The attribute \attribute{prov:type} must be set to the name of the specialized relation, e.g. \attribute{voprov:hadDescription} or  \attribute{voprov:hasConfiguration}.

To further export some concepts of the IVOA model to equivalent concepts in the W3C model, the following rules may be respected:

\begin{itemize}
\item attribute \texttt{voprov:name} $\rightarrow$ \texttt{prov:label}
\item attribute \texttt{voprov:annotation} $\rightarrow$ \texttt{prov:description}
%\item attribute \texttt{prov:role} is not allowed in W3C's \class{wasAttributedTo}, thus use \texttt{voprov:role}
%\item \class{hadMember} has no id and no optional attributes in W3C
%\item Specialized entities: add \texttt{prov:type = 'voprov:...'}
%\item Specialized relations: add \texttt{prov:type = 'voprov:...'}
\item \class{voprov:ActivityDescription}: also add \texttt{prov:type = 'prov:Plan'}
\item \class{voprov:hadDescription}: replace by \class{prov:wasAssociatedWith} with the activityDescription given as the plan through the \class{prov:hadPlan} relation
\end{itemize}

%In the case of the \class{hadDescription} relation to \class{ActivityDescription}, it may be written as a \class{wasAssociatedWith} relation between the corresponding activity and the activityDescription (seen as a prov:Plan), specifying or not the involved agent.


%see Section~\ref{sec:w3cserialization}.

% NOTE: we actually never use PROV-XML here!!

%example KR provn 
Here is an example of a serialization instance document for an entity being processed by an activity, in PROV-N notation:

%\begin{verbnobox}[\scriptsize]
%\begin{lstlisting}[language=sh, frame=single, escapeinside={}{}, caption= example of a PROV-N serialisation]
\begin{minted}[breaklines,breakanywhere,frame=single,fontsize=\footnotesize]{sh}
document
  prefix ivo <http://www.ivoa.net/documents/rer/ivo/>
  prefix voprov <http://www.ivoa.net/documents/dm/provdm/voprov/>
  prefix prov <http://www.w3.org/ns/prov#>
  prefix ex <http://www.example.com/provenance/>

  entity(ivo://example#Public_NGC6946, [prov:label="Processed image of NGC 6946", prov:type="voprov:Data"])
  entity(ivo://example#DSS2.143, [prov:label="Unprocessed image of NGC 6946", prov:type="voprov:Data"])
  activity(ex:Process1, 2017-04-18T17:28:00, 2017-04-19T17:29:00, [prov:label="Process 1"])
  used(ex:Process1, ivo://example#DSS2.143, -)
  wasGeneratedBy(ivo://example#Public_NGC6946, ex:Process1, 2017-05-05T00:00:00)
endDocument
\end{minted}

\noindent
Here is the same example in PROV-JSON format:

%\begin{verbnobox}[\scriptsize]
%\begin{lstlisting}[language=java, frame=single, caption= example of a PROV-JSON serialisation]
\begin{minted}[breaklines,breakanywhere,frame=single,fontsize=\footnotesize]{json}
{
  "prefix": {
    "ivo": "http://www.ivoa.net/documents/rer/ivo/",
    "voprov": "http://www.ivoa.net/documents/dm/provdm/voprov/",
    "prov": "http://www.w3.org/ns/prov#",
    "ex": "http://www.example.com/provenance/"
  },
  "activity": {
    "ex:Process1": {
      "prov:startTime": "2017-04-18T17:28:00",
      "prov:endTime": "2017-04-19T17:29:00",
      "prov:label": "Process 1"
    }
  },
  "wasGeneratedBy": {
    "_:id4": {
      "prov:time": "2017-05-05T00:00:00",
      "prov:entity": "ivo://example#Public_NGC6946",
      "prov:activity": "ex:Process1"
    }
  },
  "used": {
    "_:id1": {
      "prov:entity": "ivo://example#DSS2.143",
      "prov:activity": "hips:AlaRGB1"
    }
  },
  "entity": {
    "ivo://example#DSS2.143": {
      "prov:label": "Unprocessed image of NGC6946",
      "prov:type": "voprov:Data"
    },
    "ivo://example#Public_NGC6946": {
      "prov:label": "Processed image of NGC 6946",
      "prov:type": "voprov:Data"
    }
  }
}
\end{minted}

\noindent
PROV-JSON, PROV-N and PROV-XML can be converted into each other, e.g.  using the \texttt{prov} or \texttt{voprov} python package (see Section ``voprov'' in Implementation Note \citep{std:ProvenanceImplementationNote}).

\subsection{VOTable format for Provenance metadata} 

%\TODO{Move this section to ProvTAP document?}

To emphasize the compatibility to the IVOA framework, where the XML-based VOTable
format is a reference to circulate metadata, we define a VOTable mapping
specification. All classes' declarations and relations described for this data model are translated into separated tables, one for each class of the model.
% , also see the ProvTAP document.
% see Appendix~\ref{sec:appendix-prov-votable}.
All attributes of these classes are
translated to columns, i.e. VOTable FIELDS. In addition, the specification
defines the VOTable values of the FIELD and PARAM attributes \texttt{ucd},
\texttt{datatype}, \texttt{utype}, \texttt{unit}, \texttt{description}, etc.

This can be appropriately used for two goals:
\begin{itemize}
	\item Publishing full provenance metadata for data collections in VOTable format. This can be produced by data processing workflows or as output of databases containing provenance metadata.
	\item Providing the backbone for the TAP schema describing IVOA provenance metadata which is used for ProvTAP 
\end{itemize}

These VOTable serializations can be produced using the \texttt{voprov} python module, available to the community, as mentionned in see also in Implementation Note \citep[]{std:ProvenanceImplementationNote}.

Here is a VOTable document transcription of the serialization example given above in PROV-N and PROV-JSON:

%\begin{verbnobox}[\scriptsize]
%\begin{lstlisting}[language=XML, frame=single, caption= example of a VOTable serialisation]

%\listingcaption{Prov-N example 1 }

\begin{minted}[breaklines,breakanywhere,frame=single,fontsize=\footnotesize]{xml}

<?xml version="1.0" encoding="UTF-8"?>
<VOTABLE version="1.2" xmlns="http://www.ivoa.net/xml/VOTable/v1.2" 
  xmlns:ex="http://www.example.com/provenance" 
  xmlns:ivo="http://www.ivoa.net/documents/rer/ivo/"
  xmlns:voprov="http://www.ivoa.net/documents/dm/provdm/voprov/" 
  xmlns:xsi="http://www.w3.org/2001/XMLSchema-instance" 
  xsi:schemaLocation="http://www.ivoa.net/xml/VOTable/v1.2 http://www.ivoa.net/xml/VOTable/VOTable-1.2.xsd">
  <RESOURCE type="provenance">
    <DESCRIPTION>Provenance VOTable</DESCRIPTION>
    <TABLE name="Used" utype="voprov:Used">
      <FIELD arraysize="*" datatype="char" name="u_activity_id" ucd="meta.id" utype="voprov:Used.activity"/>
      <FIELD arraysize="*" datatype="char" name="u_entity_id" ucd="meta.id" utype="voprov:Used.entity"/>
      <DATA>
        <TABLEDATA>
          <TR>
            <TD>ex:Process1</TD>
            <TD>ivo://example#DSS2.143</TD>
          </TR>
        </TABLEDATA>
      </DATA>
    </TABLE>
    <TABLE name="WasGeneratedBy" utype="voprov:wasGeneratedBy">
      <FIELD arraysize="*" datatype="char" name="wgb_entity_id" ucd="meta.id" utype="voprov:WasGeneratedBy.entity"/>
      <FIELD arraysize="*" datatype="char" name="wgb_activity_id" ucd="meta.id" utype="voprov:WasGeneratedBy.activity"/>
      <DATA>
        <TABLEDATA>
          <TR>
            <TD>ivo://example#Public_NGC6946</TD>
            <TD>ex:Process1</TD>
          </TR>
        </TABLEDATA>
      </DATA>
    </TABLE>
    <TABLE name="Activity" utype="voprov:Activity">
      <FIELD arraysize="*" datatype="char" name="a_id" ucd="meta.id" utype="voprov:Activity.id"/>
      <FIELD arraysize="*" datatype="char" name="a_name" ucd="meta.title" utype="voprov:Activity.name"/>
      <FIELD arraysize="*" datatype="char" name="a_startTime" ucd="time.start" utype="voprov:Activity.startTime"/>
      <FIELD arraysize="*" datatype="char" name="a_endTime" ucd="time.end" utype="voprov:Activity.endTime"/>
      <DATA>
        <TABLEDATA>
          <TR>
            <TD>ex:Process1</TD>
            <TD>Process 1</TD>
            <TD>2017-04-18 17:28:00</TD>
            <TD>2017-04-19 17:29:00</TD>
          </TR>
        </TABLEDATA>
      </DATA>
    </TABLE>
    <TABLE name="Entity" utype="voprov:Entity">
      <FIELD arraysize="*" datatype="char" name="e_id" ucd="meta.id" utype="voprov:Entity.id"/>
      <FIELD arraysize="*" datatype="char" name="e_name" ucd="meta.title" utype="voprov:Entity.name"/>
      <FIELD arraysize="*" datatype="char" name="e_type" ucd="meta.main" utype="voprov:Entity.type"/>
      <DATA>
        <TABLEDATA>
          <TR>
            <TD>ivo://example#DSS2.143</TD>
            <TD>Unprocessed image of NGC6946</TD>
            <TD>Data</TD>
          </TR>
          <TR>
            <TD>ivo://example#Public_NGC6946</TD>
            <TD>Processed image of NGC 6946</TD>
            <TD>Data</TD>
          </TR>
        </TABLEDATA>
      </DATA>
    </TABLE>
    <INFO name="QUERY_STATUS" value="OK"/>
  </RESOURCE>
</VOTABLE>
\end{minted}

This VOTable serialization can be considered as a flat view on the various tables stored in a database implementing the data model structure explained in Section~\ref{sec:datamodel}.
More examples of serialization documents are provided in Appendix \ref{sec:appendix-serialization-examples}.
It is possible to create separate tables for each specialized entity, providing that the TABLE tag has the name of this specialized entity, and \attribute{utype=voprov:Entity}.

Such serializations can be retrieved through IVOA access protocols (see Section~\ref{sec:access_protocols}), %\ref{sec:access_protocols})
or directly integrated in dataset headers or ``associated metadata'' in order to provide provenance metadata for these datasets. 

For FITS files, a provenance extension called ``PROVENANCE'' could be added which contains provenance information of the activities that generated the FITS file. This information could be stored directly using one of the serialization formats, for example as a unique cell in an ASCII TABLE extension.

%\TODO{Add reference(s) to access protocols!}

% I believe the PROV-keyword was already used in FITS, but not PROVENANCE.
%\TODO{SVOM strategy to incorporate provenance as an extension in FITS ? still valid ?}
