\subsection{Requirements for Provenance and Use Cases}
\subsubsection{Requirements}\label{sec:requirements}

An IVOA provenance data model should provide solutions to the following tasks:

\paragraphlb{A: Tracking the production history}
        Find out which steps were taken to produce a dataset and list the methods/tools/software that was involved. 
        Track the history back to the raw data files/raw images, show the workflow.

        \noindent Examples: 
        \begin{itemize}
            \item Is the image from catalogue xxx already calibrated?
What about dark field subtraction? Were foreground stars removed? Which technique
was used?  
            
            \item Is the background noise of atmospheric muons still present in my neutrino data sample?  
        \end{itemize}

        We don't go so far as to consider easy reproducibility as a use case -- this would be too ambitious. But at least the 
        major steps undertaken to create a piece of data should be recoverable.

      
\paragraphlb{B: Attribution and further information}
        Find the people involved in the production, the people/organizations/institutes that need to be cited or can be asked for more information.

        \noindent Examples: 
        \begin{itemize}
            \item I want to use an image for my own work -- who was involved in
creating it? Whom do I need to cite or who can I contact to get this information?  
            \item I have a question about column xxx in the data
table. Who can I ask about that?  
        \end{itemize}
      

\paragraphlb{C: Aid in debugging}
        Find possible error sources.

        \noindent Examples:
        \begin{itemize}
            \item I found something strange in an image. Where does
the image come from? Which instrument was used, with which characteristics
etc.? Was there anything strange noted when the image was taken?  
            \item Which pipeline version was used -- the old one
with a known bug for treating bright objects or a newer version?  
            \item This light curve doesn't look quite right. How was
the photometry determined for each data point?  
        \end{itemize}


\paragraphlb{D: Quality assessment}
        Judge the quality of an observation, production step or data set.
        
        \noindent Examples:
        \begin{itemize}
            \item Since wrong calibration images may increase the
number of artifacts on an image rather than removing them, the knowledge about
the calibration image set will help to assess the quality of the calibrated
image.  
        \end{itemize}
      

\paragraphlb{E: Search in structured provenance metadata}
        Find all images produced by a certain processing step and similar tasks.
        
        \noindent Examples:
        \begin{itemize}
            \item Give me more images that were produced using the
same pipeline.  
            \item Give me an overview on all images reduced with the same calibration data set.  
            \item Are there any more images attributed to this observer?  
            \item Which images of the crab nebula are of good quality and were produced within the last 10 years by someone not from ESO or NASA?  
        \end{itemize}

        This task is probably the most challenging. It also includes tracking the history of data items as in A, but we still have listed this task separately, since we may decide that we can't keep this one, but we definitely want A.


\subsubsection{More specific use cases}
More specific use cases with example serialisations for different types of astronomical data sets are given in Section \ref{sec:usecases-implementations}.
