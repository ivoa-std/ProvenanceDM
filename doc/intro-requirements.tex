\subsection{Requirements and best practices}
\label{sec:requirements}

\subsubsection{Model requirements}

This document was developed with these general requirements in mind:

\begin{itemize}

% == other models / serialisation

\item Provenance information must be formalized following a standard model, with corresponding standard serialization formats.

\item Provenance information must be machine readable.

\item Provenance data model classes and attributes should make use of existing IVOA standards.

\item Provenance information should be serializable into the W3C provenance standard formats (PROV-N, PROV-XML, PROV-JSON) with minimum information loss.

\item Entities, Activities and Agents must be uniquely identifiable within a domain.
% Provenance information can only be given for uniquely identifiable entities, at least inside their domain.
\end{itemize}


\subsubsection{Best practices}

The following requirements concern the provenance usages in the VO context:

\begin{itemize}

% == links between entity/activity

\item The reliability of provenance information should be ensured (e.g., by an authority endorsing the information, or by provenance of provenance).

\item Provenance metadata for a given entity should contain information to find immediate progenitor(s).
%All produced entities must contain information to find its immediate progenitor(s).

\item An entity should be linked to the activity that generated it.
%Provenance metadata must contain information to find the activity that generated a given entity.
%* All produced entities must contain information to find the activity that generated it

\item Activities should be linked to input entities.
%(if applicable).

\item Activities should point to output entities.

\item Provenance information should make it possible to derive the logical sequence of activities.
%The order of the activities should be available.

%\item Provenance information should contain the list of activities and progenitor entities.
% too vague .... must be an ordered list ... One step should also be allowed.

%\item Released entities SHOULD have a main contact.
% same as below.

\item All activities and entities are recommended to have contact information and contain a (short) description or link to a description.
% could also be the documentation.

\end{itemize}

