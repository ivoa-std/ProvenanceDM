\section{Discussion}

\subsection{Links, ids}\label{sec:links_between_data}
It would be convenient, if each data object or even each file 
gets a unique id that can be referenced. The W3C provenance model requires ids
for entities, activities and agents, and they have to be qualified strings, 
i.e. containing a namespace. For example, an activity in the RAVE-pipeline could 
have the id `\texttt{rave:radialvelocity\_pipeline\_20160901}'. Using a namespace for each 
project for these ids will help to make them unique. 

If several copies of a dataset exist, and one of them is corrupted, it would even be useful to know
exactly which copy was used by a given activity. This can be modeled already 
with the existing tools (using a copy-activity), but we doubt that many people
would actually need this level of detail.

IVOIDs and DOI's are potentially good candidates for unique identifiers.


%\subsubsection{Calibration data}
%The calibration dataset consists of images that can be used to calibrate the
%raw data. It is not necessary to mention them explicitly in the model, 
%they are just another dataset that is used by activities with a 
%calibration-method.

%\subsubsection{Quality}
%For expressing the quality of data, we could simply define additional 
%attributes for each \class{Activity}
%or \class{Entity} object, i.e. zero, one, or more properties in the form of
%key-value pairs. We could use a \class{Quality} namespace to mark a keyword
%as quality-related:
%\begin{itemize}
%    \item quality:comment: [some text]
%    \item quality:seeing: [some value]
%\end{itemize}
%The values could range from a float number to free text.


%\subsubsection{Provenance of provenance}
%``Bundles'' are used to name a set of provenance descriptions. It is a type for 
%an entity, and allows to express provenance of provenance. This is probably  
%very interesting for workflow systems.
% -- partially covered already with ActivityFlow

\subsection{Discussion of description side}
This model was established with mainly having a database implementation in mind. 
However, it may be better in the long run to store provenance with 
the entities themselves, e.g. as an additional extension in fits-headers.

A model using a description side for defining templates for activities and
entities has the advantage of normalisation: the common processes could be 
described once and for all at some place and then be reused when describing 
the actual provenance of certain entities and activities. This \emph{some place} 
is actually the crucial point here.
In an ideal world, ``some place'' could collect all the descriptions from all 
the possible datasets and methods in astronomy, but building such a look-up place 
is a quite challenging task -- it will probably never be complete. There's also 
the issue of persistent identifiers/broken links to consider.
Normalisation is useful for closed systems, e.g. for describing the provenance 
for data produced by a certain pipeline (e.g. MuseWise system) or with 
workflow tools or when a task needs to be repeated many times. However, the VO 
is quite the contrary of a closed system and we need to keep an eye on what is 
actually achievable.

When writing down a simple serialisation of e.g. the provenance for a stacked 
image with the protoype-model, it soon becomes quite cumbersome to define 
everything twice: first the descriptions, then the instances. This basically 
doubles the number of entries to describe provenance (unless there is already 
some place with all the descriptions to which we can refer).

Expressing provenance for a stacked image with this smaller set of classes may 
be simpler, but on the other hand constructing a database schema becomes much 
harder. 
We could leave it to the implementors to choose what is more useful for them, 
and when extracting provenance, serialising it, then the descriptions are 
combined with the activity/entity for 
the serialisation, thus probably producing some repetition, but avoiding too 
many links between different items.

%\Note{Descriptions could be present in W3C-conform serialisations, if we 
%put them into entities.}

%\TODO{Check, if PROV-Templates from the W3C (inofficial note) could be used 
%for ActivityDescriptions.}

\subsection{Discussion of ActivityFlow}
By introducing the \class{ActivityFlow} class, one entity can now have many 
wasGeneratedBy-links to activities. One of them would be the actual generation-activity, 
the other activities can only be activityflows containing this generation-activity.
This is not expressed explicitely in the current model. 

We could introduce an additional abstract class, e.g. \class{AbstractActivity}, with \class{Activity} and 
\class{ActivityFlow} being subclasses to this one. But this adds another layer of complexity 
that we may not want in this data model.

Since we introduced \class{ActivityFlow} mainly for having different view levels, 
we may want to add an attribute \emph{viewLevel} to descriptions of activityflows.

We are planning to test how it all works in implementations and will then adjust the model 
accordingly.

\subsection{VO-DML representation}
We do not yet have a VO-DML compliant representation of the model. This is one 
of the issues to be discussed for the next version of this model.

\subsection{Links to other data models}
Section~\ref{sec:dmlinks} still needs to be expanded further, especially making detailed links with the 
Simulation Data Model will be very useful.
