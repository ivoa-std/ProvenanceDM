\subsubsection{ProvDAL}
ProvDAL is a service the interface of which is organized around one main parameter, the ``ID'' of an entity (obs\_publisher\_did of an ObsDataSet for example) or activity. The response is given in one of the following formats: PROV-N, PROV-JSON, PROV-XML, PROV-VOTABLE. Additional parameters can complete ID to refine the query: FORMAT allows to choose the output format. BACKWARD gives the number of relations that shall be tracked in backward direction, i.e. along the provenance history. Its value is either 0, a positive integer or ALL. If this parameter is omitted, the default is ALL, wich returns the complete provenance history.
The optional parameter FORWARD defines the number of forward relations; it's also either a positive integer or ALL, but default is 0. That means if neither FORWARD nor BACKWARD are specified, then the complete provenance history is returned.


%\comment{Maybe it's better to use DEPTH and DIRECTION instead of FORWARD and BACKWARD. Reason: if a service just implements the backward direction, then it's weird to call something ``backward'' if there is no ``forward'' as well. DEPTH is also a commonly used word when refering to graphs and numbers of relations.}

The ID parameter is allowed more than once in order to retrieve several data set provenance details at the same time. An example request could look like this:

\begin{verbatim}
{provdal-base-url}?ID=rave:dr4&BACKWARD=1&FORMAT=PROV-JSON
\end{verbatim}

Each of the provenance relation has a direction, BACKWARD follows these directions whereas FORWARD follows the relations in reverse direction, independent of the relation type. This is easier to implement, but has the (for a user unexpected) side effect that e.g. agent relations are only retrieved when using BACKWARD, but never with FORWARD. Similarly for membership (hadStep, hadMember) relations: members of a collection or activityFlow are retrieved only in BACKWARD direction, and collections or activityFlows that contain an entity or activity are only found in FORWARD direction. In order to provide a more user-friendly interface with less surprising behaviour, we define three more request parameters: EXPAND\_AGENT, EXPAND\_COLLECTION and EXPAND\_ACTIVITYFLOW. They take TRUE or FALSE as arguments. If they are set to TRUE, the relations with agents, collections and activityFlows will be included in any case, independent of the direction in which the provenance graph is retrieved.
\TODO{Draw a provenance graph picture here with different relation types and arrows for direction.}
\TODO{Implementations need to show if this is really the best way.}

\TODO{If EXPAND\_AGENT=TRUE: include all agent relations, but if EXPAND\_AGENT=FALSE, then use default behaviour? Or do not include any of the agent relations? Which one would it be?}


A ProvDAL service MUST implement the parameters ID, BACKWARD and FORMAT; the remaining parameters are optional.
 If a service does not implement the optional parameters, but they appear in the request, then the service should return with an error.

Table~\ref{tab:provdal-parameters} summarizes the parameters for such a ProvDAL service interface.

\begin{table}[h]
\small
\begin{tabulary}{1.0\textwidth}{@{}p{0.17\textwidth}Lp{0.2\textwidth}p{0.10\textwidth}p{0.3\textwidth}@{}}
%{llp{0.2\textwidth}p{0.3\textwidth}}
\toprule
\head{Parameter} & \head{Requirement} & \head{Value/options} & \head{Default} & \head{Description}\\\hline
\midrule
ID & required & qualified ID & -- & a valid qualified identifier for an entity or activity (can occur multiple times)\\
BACKWARD & required & 0,1,2,..., ALL & ALL & number of relations to be followed backwards or \texttt{ALL} for everything\\
FORWARD & optional & 0,1,2,..., ALL & 0 & number of relations to be followed forward or \texttt{ALL} for everything\\
FORMAT & required & PROV-N, PROV-JSON, PROV-XML, PROV-VOTABLE & ? & serialisation format of the response\\
EXPAND\_ AGENT & optional & TRUE or FALSE & TRUE & include agent relations in any case\\
EXPAND\_ COLLECTION & optional & TRUE or FALSE & TRUE & include relations with collections in any case\\
EXPAND\_ ACTIVITYFLOW & optional & TRUE or FALSE & TRUE & include relations with activityFlows in any case\\
\bottomrule
\end{tabulary}
\caption{ProvDAL request parameters}
\label{tab:provdal-parameters}
\end{table}


