\subsection{Introduction, serialization formats}
\label{sec:intro-serialization}
Serialization files constitute the building blocks of the client/server and client/client dialogs.
The provenance information as represented in the data model is split in three main concepts that can be searched following many different relations between the main 3 classes, \class{Activity}, \class{Entity} and \class{Agent}.
The selection of the relations to expose when distributing the provenance information depends on the usage and will be described more extensively in the use cases in Section~\ref{sec:usecases-implementations}, the Implementation Note \citep{std:ProvenanceImplementationNote} and the links therein.

To give a very simple example, suppose a client asks for the context of execution for one specified activity, which computes a simple RGB color composition. 
On the server side, exposing the provenance information for this activity or for an entity, corresponding to a monocolor or RGB image, means to just expose the structure of the classes and relation tables and feed them with the related tuples in the database.
On the client side, the content of a VO-Provenance serialization document can then be explored and represented using graphical interfaces, as inspired by the Provenance Southampton suite or by customized visualization tools.

\subsection{Serialization formats: PROV-N, PROV-JSON and PROV-XML} % from W3C reused and extended
The serialization formats {PROV-N}, {PROV-JSON} and {PROV-XML} are proposed 
in the W3C Provenance framework for storing and exchanging the provenance metadata: {PROV-N}, {PROV-JSON} and {PROV-XML}, defined in \cite{std:PROV-N}, \cite{std:PROV-JSON} and \cite{std:PROV-XML}, respectively. 
%These are serializations of the W3C provenance data model, a larger set of classes and relations compared to this model but sharing the same core structure. 
They can be reused here as well for serializations of our data model. For producing fully W3C compatible serializations, see Section~\ref{sec:w3cserialization}.

% NOTE: we actually never use PROV-XML here!!

%example KR provn 
Here is an example serialization instance document for an entity being processed by an activity, in PROV-N format:

\begin{verbnobox}[\scriptsize]

document
  prefix ivo <http://www.ivoa.net/documents/rer/ivo/>
  prefix ex <http://www.example.com/provenance/>
  prefix voprov <http://www.ivoa.net/documents/dm/provdm/voprov/>

  entity(ivo://example#Public_NGC6946, [voprov:name="Processed image of NGC 6946"])
  entity(ivo://example#DSS2.143, [voprov:name="Unprocessed image of NGC 6946"])
  activity(ex:Process1, 2017-04-18T17:28:00, 2017-04-19T17:29:00, [voprov:name="Process 1"])
  used(ex:Process1, ivo://example#DSS2.143, -)
  wasGeneratedBy(ivo://example#Public_NGC6946, ex:Process1, 2017-05-05T00:00:00)
endDocument

\end{verbnobox}

%example KR provjson (entity, agent and activity)

\begin{verbnobox}[\scriptsize]
{
  "prefix": {
    "ivo": "http://www.ivoa.net/documents/rer/ivo/",
    "voprov": "http://www.ivoa.net/documents/dm/provdm/voprov/",
    "ex": "http://www.example.com/provenance/"
  },
  "activity": {
    "ex:Process1": {
      "voprov:startTime": "2017-04-18T17:28:00",
      "voprov:endTime": "2017-04-19T17:29:00",
      "voprov:name": "Process 1"
    }
  },
  "wasGeneratedBy": {
    "_:id4": {
      "voprov:time": "2017-05-05T00:00:00",
      "voprov:entity": "ivo://example#Public_NGC6946",
      "voprov:activity": "ex:Process1"
    }
  },
  "used": {
    "_:id1": {
      "voprov:entity": "ivo://CDS/P/DSS2/POSSII#POSSII.J-DSS2.143",
      "voprov:activity": "hips:AlaRGB1"
    }
  }
  "entity": {
    "ivo://example#DSS2.143": {
      "voprov:name": "Unprocessed image of NGC6946"
    },
    "ivo://example#Public_NGC6946": {
      "voprov:name": "Processed image of NGC 6946"
    }
  }
}
\end{verbnobox}

\subsection{PROV-VOTable format} 
To emphasize the compatibility to the IVOA framework, where the VOTable-XML
format is a reference to circulate metadata, we define a PROV-VOTable mapping
specification. All classes' declarations and relations described in PROV-N are
translated into separated tables, one for each class of the model, see
Appendix~\ref{sec:appendix-prov-votable}. All attributes of these classes are
translated to columns, i.e. VOTable FIELDS. In addition, the specification
defines the VOTable values of the FIELD and PARAM attributes \texttt{ucd},
\texttt{datatype}, \texttt{utype}, \texttt{unit}, \texttt{description}, etc. 

This can be appropriately used for two goals:
\begin{itemize}
	\item Publishing full provenance metadata for data collections in VOTable format. This can be produced by data processing workflows or as output of databases containing provenance metadata.
	\item Providing the backbone for the TAP schema describing IVOA provenance metadata which isused for ProvTAP 
\end{itemize}

These VOTable serializations can be produced using the voprov\footnote{\url{https://github.com/sanguillon/voprov}} python module, available to the community, see also Section~\ref{sec:implementation_voprov} and the IVOA ProvenanceDM Implementation Note \citep[]{std:ProvenanceImplementationNote}. 

%example KR prov VOTable .
Here is a VOTable document transcription of the serialization example given above in PROV-N and PROV-JSON:

\begin{verbnobox}[\scriptsize]
<?xml version="1.0" encoding="UTF-8"?>
<VOTABLE version="1.2" xmlns="http://www.ivoa.net/xml/VOTable/v1.2" xmlns:ex="http://www.example.com/provenance" xmlns:ivo="http://www.ivoa.net/documents/rer/ivo/" xmlns:voprov="http://www.ivoa.net/documents/dm/provdm/voprov/" xmlns:xsi="http://www.w3.org/2001/XMLSchema-instance" xsi:schemaLocation="http://www.ivoa.net/xml/VOTable/v1.2 http://www.ivoa.net/xml/VOTable/VOTable-1.2.xsd">
  <RESOURCE type="provenance">
    <DESCRIPTION>Provenance VOTable</DESCRIPTION>
    <TABLE name="Usage" utype="voprov:used">
      <FIELD arraysize="*" datatype="char" name="activity" ucd="meta.id" utype="voprov:Usage.activity"/>
      <FIELD arraysize="*" datatype="char" name="entity" ucd="meta.id" utype="voprov:Usage.entity"/>
      <DATA>
        <TABLEDATA>
          <TR>
            <TD>ex:Process1</TD>
            <TD>ivo://example#DSS2.143</TD>
          </TR>
        </TABLEDATA>
      </DATA>
    </TABLE>
    <TABLE name="Generation" utype="voprov:wasGeneratedBy">
      <FIELD arraysize="*" datatype="char" name="entity" ucd="meta.id" utype="voprov:Generation.entity"/>
      <FIELD arraysize="*" datatype="char" name="activity" ucd="meta.id" utype="voprov:Generation.activity"/>
      <DATA>
        <TABLEDATA>
          <TR>
            <TD>ivo://example#Public_NGC6946</TD>
            <TD>ex:Process1</TD>
          </TR>
        </TABLEDATA>
      </DATA>
    </TABLE>
    <TABLE name="Activity" utype="voprov:Activity">
      <FIELD arraysize="*" datatype="char" name="id" ucd="meta.id" utype="voprov:Activity.id"/>
      <FIELD arraysize="*" datatype="char" name="name" ucd="meta.title" utype="voprov:Activity.name"/>
      <FIELD arraysize="*" datatype="char" name="start" ucd="" utype="voprov:Activity.startTime"/>
      <FIELD arraysize="*" datatype="char" name="stop" ucd="" utype="voprov:Activity.endTime"/>
      <DATA>
        <TABLEDATA>
          <TR>
            <TD>ex:Process1</TD>
            <TD>Process 1</TD>
            <TD>2017-04-18 17:28:00</TD>
            <TD>2017-04-19 17:29:00</TD>
          </TR>
        </TABLEDATA>
      </DATA>
    </TABLE>
    <TABLE name="Entity" utype="voprov:Entity">
      <FIELD arraysize="*" datatype="char" name="id" ucd="meta.id" utype="voprov:Entity.id"/>
      <FIELD arraysize="*" datatype="char" name="name" ucd="meta.title" utype="voprov:Entity.name"/>
      <DATA>
        <TABLEDATA>
          <TR>
            <TD>ivo://example#DSS2.143</TD>
            <TD>Unprocessed image of NGC6946</TD>
          </TR>
          <TR>
            <TD>ivo://example#Public_NGC6946</TD>
            <TD>Processed image of NGC 6946</TD>
          </TR>
        </TABLEDATA>
      </DATA>
    </TABLE>
    <INFO name="QUERY_STATUS" value="OK"/>
  </RESOURCE>
</VOTABLE>

\end{verbnobox}

This VOTable serialization can be considered as a flat view on the various tables stored in a database implementing the datamodel structure explained in Section~\ref{sec:datamodel}.
More examples of serialization documents are provided in Appendix \ref{sec:appendix-serialization-examples}.


Such serializations can be retrieved through access protocols (see \ref{sec:access_protocols} ) or directly integrated in dataset headers or ``associated metadata'' in order to provide provenance metadata for these datasets. E.g. for FITS files a provenance extension called ``PROVENANCE'' could be added which contains provenance information of the workflow that generated the FITS file in one of the serialization formats.

% I believe the PROV-keyword was already used in FITS, but not PROVENANCE.
\TODO{SVOM strategy to incorporate provenance as an extension in FITS ? still valid ?}


\subsection{Serialization of description classes in the data processing context}
\label{sec:description-serialization}

%{updated by Mathieu} 
%The ProvenanceDM includes description classes that can exist before any provenance information is recorded. 
Description classes in the Provenance DM gather information
 on the data processing which can preexist to the processing itself.
First, the ActivityDescription class gives generic information on the activity (name, description, doculink...) and the parameters expected as an input. In addition, UsedDescription and WasGeneratedByDescription classes indicate the expected roles of the input and output entities respectively. Finally, The activity may expect specific kinds of entities as inputs or outputs, for which there may be detailed descriptions stored as EntityDescription records.

The serialization of an ActivityDescription, that includes all those description classes, is based on the IVOA DataLink Service Descriptors for service resources \citep{std:Datalink}, and can thus be stored as a VOTable  \citep{std:VOTABLE}. Indeed, a service descriptor points to a service that probably executes an activity using the given input parameters, some of which probably point to entities. One can thus easily translate an ActivityDescription VOTable to a DataLink service descriptor VOTable block, and vice-versa. 

The VOTable contains one resource with attributes type=``meta'' and utype=``voprov:ActivityDescription''. This resource contains PARAM elements to describe the activity and then GROUP elements gathering additional PARAM elements to describe:
\begin{itemize}
 \item the input parameters (group name=``InputParams''),
 \item the input entities (group name=``Used''),
 \item the output entities (group name=``Generated''). 
 \end{itemize} 
% comment Mireille: 
% what if an activity computes a number , a boolean as a result and produces an output parameter instead of an output entity: e.g compare 2 sets of dataproducts : false/True correlate two datasets : correlation value, etc ...  
% --> add output parameters  ?

The standard PARAM elements for an activity resource correspond to the attributes of the ActivityDescription class (see Section~\ref{sec:activity}) and may include an Agent name and email. The utype attribute is used to connect the attribute to its location in the Provenance DM.
For the input parameters, each ParameterDescription attribute is mapped to a PARAM attribute (both have the same structure, see Section \ref{sec:parameters}).
% suggestion Mireille 
% change ``attribute'' to ``element'' to be compliant to the Votable definitions : PARAM, FIELDS are XML elements in the VOtble schema , utype, datatype , etc ... are XML attribute  
% suggestion below 
% ++ The utype attribute in VOTable is used to connect the VOTable element ( PARAM or FIELD) to its location in the Provenance DM.
%++ For the input parameters, each ParameterDescription attribute in the model is mapped to a PARAM element in the VOTable (both have the same structure, see Section \ref{sec:parameters}).

%
For the input and output entity groups, each related entity is described with a GROUP block that contains several PARAM elements with those names:
\begin{itemize}
 \item name="id" for the default identifier of the entity (e.g. a file name),
 \item name="role" that gives the role of the entity with respect to the Used or WasGeneratedBy relation (e.g. "red", "green" or "blue" channel image, or the output "RGB" file),
 \item name="content\_type" for the MIME type expected by the activity for the input or output entity,
 \item other additional PARAM elements corresponding to the EntityDescription attributes.
 \end{itemize} 
% mireille suppressed one 'expected'in sentence above
It is possible to reference an input parameter using the \emph{ref} attribute of PARAM, if an input or output entity is referenced as an input parameter to the activity (e.g. the name of an input file). In the following example the output file name "RGB.jpg" is expected as a parameter to the activity and is thus referenced by the id PARAM element located in the GROUP of generated entities (with ref="RGB").

Here is an example of an ActivityDescription VOTable that describes an activity to create an RGB image from three red, green, blue images:
%++ suggestion mireille : 
%Here is an example of an ActivityDescription VOTable that describes an activity to create an RGB image from three input images mapped to the red, green, blue image planes in the composition. 

\begin{verbnobox}[\scriptsize]

<VOTABLE xmlns:xsi="http://www.w3.org/2001/XMLSchema-instance" 
    xmlns="http://www.ivoa.net/xml/VOTable/v1.3" version="1.3" 
    xsi:schemaLocation="http://www.ivoa.net/xml/VOTable/v1.3 
    http://www.ivoa.net/xml/VOTable/v1.3">
    
  <RESOURCE ID="make_RGB_image" name="make_RGB_image" 
      type="meta" utype="voprov:ActivityDescription">
      
    <DESCRIPTION>Create an RGB image from 3 images</DESCRIPTION>
    <LINK content-role="doc" href="..." />
    <PARAM name="name" datatype="char" arraysize="*" 
        value="make_RGB_image" utype="voprov:ActivityDescription.label" />
    <PARAM name="type" datatype="char" arraysize="*" 
        value="None" utype="voprov:ActivityDescription.type"/>
    <PARAM name="subtype" datatype="char" arraysize="*" 
        value="None" utype="voprov:ActivityDescription.subtype" />
    <PARAM name="version" datatype="float" 
        value="None" utype="voprov:ActivityDescription.version" />
    <PARAM name="contact_name" datatype="char" arraysize="*" 
        value="..." utype="voprov:Agent.name" />
    <PARAM name="contact_email" datatype="char" arraysize="*" 
        value="...@..." utype="voprov:Agent.email" />
        
    <GROUP name="InputParams" utype="voprov:Parameter">
      <PARAM ID="RGB" arraysize="*" datatype="char" name="RGB" 
          type="no_query" value="RGB.jpg">
        <DESCRIPTION>RGB image name</DESCRIPTION>
      </PARAM>
      <PARAM ID="order" arraysize="*" datatype="char" name="order" 
          type="no_query" value="RGB">
        <DESCRIPTION>order of the channels</DESCRIPTION>
        <VALUES>
          <OPTION value="RGB"/>
          <OPTION value="RBG"/>
          <OPTION value="GBR"/>
          <OPTION value="GRB"/>
          <OPTION value="BRG"/>
          <OPTION value="BGR"/>
        </VALUES>
      </PARAM>
    </GROUP>
    %mir question : why do we need the order description here?once we know each role , there will be only one color composition the R.jpg will enter the red channel. the association is made at the level of entity--> entity.desc mapping ? 
		% not sure I get it right here , sorry.
		
    <GROUP name="Used" utype="voprov:Used">
      <GROUP name="R" utype="voprov:Entity">
        <DESCRIPTION>Image for red channel</DESCRIPTION>
        <PARAM arraysize="*" datatype="char" name="id" value="R.jpg" utype="voprov:Entity.id" />
        <PARAM arraysize="*" datatype="char" name="role" value="red" utype="voprov:Used.role" />
        <PARAM arraysize="*" datatype="char" name="content_type" value="image/jpeg" utype="voprov:EntityDescription.content_type" />
      </GROUP>
      <GROUP name="G" utype="voprov:Entity">
        <DESCRIPTION>Image for green channel</DESCRIPTION>
        <PARAM arraysize="*" datatype="char" name="id" value="G.jpg" utype="voprov:Entity.id" />
        <PARAM arraysize="*" datatype="char" name="role" value="green" utype="voprov:Used.role" />
        <PARAM arraysize="*" datatype="char" name="content_type" value="image/jpeg" utype="voprov:EntityDescription.content_type" />
      </GROUP>
      <GROUP name="B" utype="voprov:Entity">
        <DESCRIPTION>Image for blue channel</DESCRIPTION>
        <PARAM arraysize="*" datatype="char" name="id" value="B.jpg" utype="voprov:Entity.id" />
        <PARAM arraysize="*" datatype="char" name="role" value="blue" utype="voprov:Used.role" />
        <PARAM arraysize="*" datatype="char" name="content_type" value="image/jpeg" utype="voprov:EntityDescription.content_type" />
      </GROUP>
    </GROUP>
    
    <GROUP name="Generated" utype="voprov:WasGeneratedBy">
      <GROUP name="RGB" utype="voprov:Entity">
        <DESCRIPTION>RGB image generated</DESCRIPTION>
        <PARAM arraysize="*" datatype="char" name="id" value="RGB.jpg" ref="RGB" utype="voprov:Entity.id" />
        <PARAM arraysize="*" datatype="char" name="role" value="RGB" utype="voprov:WasGenereratedBy.role" />
        <PARAM arraysize="*" datatype="char" name="content_type" value="image/jpeg" utype="voprov:EntityDescription.content_type" />
      </GROUP>
      </PARAM>
    </GROUP>
    
  </RESOURCE>
</VOTABLE>

\end{verbnobox}

\subsection{W3C PROV-DM compatible serializations}
According to our minimum requirements (see Section~\ref{sec:requirements}), it must be possible to
serialize the provenance metadata into the W3C compatible formats, so that it can be exchanged within a wider context and can processed by already existing tools.
In our data model we have defined additional classes and attributes that are not W3C conform and thus need to be
serialized with different names/structures. Using ``voprov'' as the namespace prefix for our model and ``prov'' for W3C PROV-DM, the necessary changes for mapping from ProvenanceDM to W3C PROV-DM are:

\begin{itemize}
\item namespace \texttt{voprov} $\rightarrow$ \texttt{prov} for those attributes that are the same in W3C (e.g. ID, role, startTime, endTime)
\item attribute \texttt{voprov:name} $\rightarrow$ \texttt{prov:label}
\item attribute \texttt{voprov:annotation} $\rightarrow$ \texttt{prov:description}
\item attribute \texttt{prov:role} is not allowed in W3C's \emph{wasAttributedTo}, thus use \texttt{voprov:role}
\item \emph{hadMember} has no ID and no optional attributes in W3C
\item \emph{Collection} $\rightarrow$ \emph{Entity} with \texttt{prov:type = prov:collection}
\item restructure \emph{ActivityFlow}:
	\begin{itemize}
	\item \emph{ActivityFlow} $\rightarrow$ \emph{Activity} with additional attribute \texttt{voprov:votype = 'voprov:activityFlow'}
	\item replace \emph{hadStep} relation by W3C's general \emph{wasInfluencedBy} relation with additional attribute \texttt{voprov:votype = 'voprov:hadStep'}
	\end{itemize}

\end{itemize}

This way, one can produce W3C compatible serializations of our model with minimum information loss.
