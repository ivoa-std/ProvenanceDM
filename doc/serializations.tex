\subsection{Introduction, serialization formats}
\label{sec:intro-serialization}
The serialization files documents constitutes the building blocks of the client/server dialogs.

The provenance information as represented in the data model is split in three main concepts that can be searched following many different relations involved between the main 3 classes. 
The selection of the relations to expose when distributing the provenance information depends on the usage and will be described more extensively in the Use-case sections (\ref{sec:usecases-implementations}) , the implementation note \citep[]{std:ProvenanceImplementationNote} and the links therein.

To give a very simple example, suppose a client asks for the context of execution for one specified Activity, which computes a simple RGB color composition. 

On the server side, exposing the Provenance information for this Activity or for an Entity, corresponding to a monocolor or RGB image, is just exposing the structure of the classes and relation tables and feed them with the related t-upples in the database.
On the client side, the content of a VO-Provenance serialization document can then be explored and represented using graphical interfaces, as inspired by the Provenance Southampton suite or by customized visualization tools.
 
\subsection{W3C Serialisation Formats reused and exented}
In the W3C Provenance framework, three descriptions formats are proposed to serialize the Provenance metadata : {PROV-N}, {PROV-JSON}, {PROV-XML} as defined in \citep[]{std:W3CProvN}, \citep[]{soton356855}. These are serializations of the W3C provenance data model, a larger set of classes and relations compared to this model but sharing the same core structure. They allow the possibility to add IVOA or \textit{ad hoc} attributes to the basic ones in each class. This way the IVOA models can produce W3C compliant serializations and take benefit of W3C visualizing tools.

%example KR provn 
Here is a serialization instance document for an entity being processed by an activity, in PROV-N format:

\begin{verbnobox}[\scriptsize]

document
  prefix ivo <http://www.ivoa.net/documents/rer/ivo/>
  prefix ex <http://www.example.com/provenance/>
  prefix voprov <http://www.ivoa.net/documents/dm/provdm/voprov/>

  entity(ivo://example#Public_NGC6946, [voprov:name="Processed image of NGC 6946"])
  entity(ivo://example#DSS2.143, [voprov:name="Unprocessed image of NGC 6946"])
  activity(ex:Process1, 2017-04-18T17:28:00, 2017-04-19T17:29:00, [voprov:name="Process 1"])
  used(ex:Process1, ivo://example#DSS2.143, -)
  wasGeneratedBy(ivo://example#Public_NGC6946, ex:Process1, 2017-05-05T00:00:00)
endDocument

\end{verbnobox}

%example KR provjson (entity, agent and activity)

\begin{verbnobox}[\scriptsize]
{
  "prefix": {
    "ivo": "http://www.ivoa.net/documents/rer/ivo/",
    "voprov": "http://www.ivoa.net/documents/dm/provdm/voprov/",
    "ex": "http://www.example.com/provenance/"
  },
  "activity": {
    "ex:Process1": {
      "prov:startTime": "2017-04-18T17:28:00",
      "prov:endTime": "2017-04-19T17:29:00",
      "voprov:name": "Process 1"
    }
  },
  "wasGeneratedBy": {
    "_:id4": {
      "prov:time": "2017-05-05T00:00:00",
      "prov:entity": "ivo://example#Public_NGC6946",
      "prov:activity": "ex:Process1"
    }
  },
  "used": {
    "_:id1": {
      "prov:entity": "ivo://CDS/P/DSS2/POSSII#POSSII.J-DSS2.143",
      "prov:activity": "hips:AlaRGB1"
    }
  }
  "entity": {
    "ivo://example#DSS2.143": {
      "voprov:name": "Unprocessed image of NGC6946"
    },
    "ivo://example#Public_NGC6946": {
      "voprov:name": "Processed image of NGC 6946"
    }
  }
}
\end{verbnobox}
\subsection{Prov-VOTable format} 
To emphasize the compatibility to the IVOA framework, where the VOTable-XML format is a reference to circulate metadata, we define a PROV-VOTABLE mapping specification. All classes' declarations and relations described in PROV-N are translated as separated tables, one for each class of the model.
All attributes of these classes are translated as columns, i.e, VOTable FIELDS. 
In addition, the specification defines the VOTable values of FIELD and PARAM attributes ucd, datatype, utype, unit, description, etc. 

This can be appropriately used for two goals:
\begin{itemize}
	\item publishing full provenance metadata for data collections in VOTable format. This can be produced by data processing workflows or as output of databases containing provenance metadata.
	\item providing the backbone for the TAP Schema describing IVOA provenance metadata which we call ProvTAP 
\end{itemize}

These VOTable serialisations can be produced  using  the VOPROV Python module \footnote{\url{https://github.com/sanguillon/voprov}} python module, available to the community. See also Section~\ref{sec:implementation_voprov} and the IVOA Prov-DM Implementation Note \citep[]{std:ProvenanceImplementationNote}. 

%example KR prov VOTABLE .
Here is VOTable document transcription of the corresponding serialization example given above in prov-N and Prov-Json.

\begin{verbnobox}[\scriptsize]
<?xml version="1.0" encoding="UTF-8"?>
<VOTABLE version="1.2" xmlns="http://www.ivoa.net/xml/VOTable/v1.2" xmlns:ex="http://www.example.com/provenance" xmlns:ivo="http://www.ivoa.net/documents/rer/ivo/" xmlns:voprov="http://www.ivoa.net/documents/dm/provdm/voprov/" xmlns:xsi="http://www.w3.org/2001/XMLSchema-instance" xsi:schemaLocation="http://www.ivoa.net/xml/VOTable/v1.2 http://www.ivoa.net/xml/VOTable/VOTable-1.2.xsd">
  <RESOURCE type="provenance">
    <DESCRIPTION>Provenance VOTable</DESCRIPTION>
    <TABLE name="Usage" utype="voprov:used">
      <FIELD arraysize="*" datatype="char" name="activity" ucd="meta.id" utype="voprov:Usage.activity"/>
      <FIELD arraysize="*" datatype="char" name="entity" ucd="meta.id" utype="voprov:Usage.entity"/>
      <DATA>
        <TABLEDATA>
          <TR>
            <TD>ex:Process1</TD>
            <TD>ivo://example#DSS2.143</TD>
          </TR>
        </TABLEDATA>
      </DATA>
    </TABLE>
    <TABLE name="Generation" utype="voprov:wasGeneratedBy">
      <FIELD arraysize="*" datatype="char" name="entity" ucd="meta.id" utype="voprov:Generation.entity"/>
      <FIELD arraysize="*" datatype="char" name="activity" ucd="meta.id" utype="voprov:Generation.activity"/>
      <DATA>
        <TABLEDATA>
          <TR>
            <TD>ivo://example#Public_NGC6946</TD>
            <TD>ex:Process1</TD>
          </TR>
        </TABLEDATA>
      </DATA>
    </TABLE>
    <TABLE name="Activity" utype="voprov:Activity">
      <FIELD arraysize="*" datatype="char" name="id" ucd="meta.id" utype="voprov:Activity.id"/>
      <FIELD arraysize="*" datatype="char" name="name" ucd="meta.title" utype="voprov:Activity.name"/>
      <FIELD arraysize="*" datatype="char" name="start" ucd="" utype="voprov:Activity.startTime"/>
      <FIELD arraysize="*" datatype="char" name="stop" ucd="" utype="voprov:Activity.endTime"/>
      <DATA>
        <TABLEDATA>
          <TR>
            <TD>ex:Process1</TD>
            <TD>Process 1</TD>
            <TD>2017-04-18 17:28:00</TD>
            <TD>2017-04-19 17:29:00</TD>
          </TR>
        </TABLEDATA>
      </DATA>
    </TABLE>
    <TABLE name="Entity" utype="voprov:Entity">
      <FIELD arraysize="*" datatype="char" name="id" ucd="meta.id" utype="voprov:Entity.id"/>
      <FIELD arraysize="*" datatype="char" name="name" ucd="meta.title" utype="voprov:Entity.name"/>
      <DATA>
        <TABLEDATA>
          <TR>
            <TD>ivo://example#DSS2.143</TD>
            <TD>Unprocessed image of NGC6946</TD>
          </TR>
          <TR>
            <TD>ivo://example#Public_NGC6946</TD>
            <TD>Processed image of NGC 6946</TD>
          </TR>
        </TABLEDATA>
      </DATA>
    </TABLE>
    <INFO name="QUERY_STATUS" value="OK"/>
  </RESOURCE>
</VOTABLE>

\end{verbnobox}

This VOTable serialization can be considered as a flat view on the various tables stored in a database implementing the datamodel structure explained in Section~\ref{sec:datamodel}.
More examples of serialization documents are provided in Appendix \ref{sec:appendix-serialization-examples}.
 

Such serializations can be retrieved through access protocols (see \ref{sec:access_protocols} ) or directly integrated in dataset headers or ``associated metadata'' in order to provide provenance metadata for these datasets. E.g. for FITS files a provenance extension called ``PROVENANCE'' could be added which contains provenance information of the workflow that generated the FITS file in one of the serialization formats.
\TODO{Check that this keyword is not already taken.}
\TODO{SVOM strategy to incorporate provenance as an extension in FITS ? still valid ?}


\subsection{Serialization of description classes}
\label{sec:description-serialization}
%{updated by Mathieu} 
The ProvenanceDM includes description classes that can exist before any provenance information is recorded. First, the ActivityDescription class gives information on the activity (name, description, doculink...) and the parameters expected as an input. In addition, UsedDescription and WasGeneratedByDescription classes indicate the expected roles of the input and output entities respectively. Finally, The activity may expect specific kinds of entities as inputs or outputs, for which there may be detailed descriptions stored as EntityDescription records.

The serialization of an ActivityDescription, that includes all those description classes, is based on the IVOA DataLink Service Descriptors for service resources \citep{std:Datalink}, and can thus be stored as a VOTable  \citep{std:VOTABLE}. Indeed, a service descriptor points to a service that probably executes an activity using the given input parameters, some of which probably point to entities. One can thus easily translate an ActivityDescription VOTable to a DataLink service descriptor VOTable block, and vice-versa. 

The VOTable contains one resource with attributes type=``meta'' and utype=``voprov:ActivityDescription''. This resource contains PARAM elements to describe the activity and GROUP elements with additional PARAM elements to describe the input parameters (group name=``InputParams''), the input entities (group name=``Used'') and the output entities (group name=``Generated''). 

The standard PARAM elements for an activity resource correspond to the attributes of the ActivityDescription class (see Section~\ref{sec:activity}) and may include an Agent name and email. For the input parameters, each ParameterDescription element is mapped to a PARAM element. The mapping is direct as ParameterDescription is based on PARAM. For the input and output entity groups, each related entity is described with a PARAM block where the name is the role of the entity in the scope of the activity, and the expected value is the entity identifier (utype=``voprov:Entity.id''). It is possible to reference an input parameter using the ref attribute of PARAM, if an input entity is given as an input parameter to the activity (e.g. the name of a file). The xtype attribute of PARAM can be used to provide the content type (MIME type) of the entity.

Here is an example of an ActivityDescription VOTable that describes an activity to create an RGB image from three red, green, blue images:


\begin{verbnobox}[\scriptsize]

<VOTABLE xmlns:xsi="http://www.w3.org/2001/XMLSchema-instance" 
    xmlns="http://www.ivoa.net/xml/VOTable/v1.3" version="1.3" 
    xsi:schemaLocation="http://www.ivoa.net/xml/VOTable/v1.3 
    http://www.ivoa.net/xml/VOTable/v1.3">
  <RESOURCE ID="make_RGB_image" name="make_RGB_image" 
      type="meta" utype="voprov:ActivityDescription">
    <DESCRIPTION>Create an RGB image from 3 images</DESCRIPTION>
    <LINK content-role="doc" href="..."/>
    <PARAM name="label" datatype="char" arraysize="*" 
        value="make_RGB_image" utype="voprov:ActivityDescription.label"/>
    <PARAM name="type" datatype="char" arraysize="*" 
        value="None" utype="voprov:ActivityDescription.type"/>
    <PARAM name="subtype" datatype="char" arraysize="*" 
        value="None" utype="voprov:ActivityDescription.subtype"/>
    <PARAM name="version" datatype="float" 
        value="None" utype="voprov:ActivityDescription.version"/>
    <PARAM name="contact_name" datatype="char" arraysize="*" 
        value="..." utype="voprov:Agent.name"/>
    <PARAM name="contact_email" datatype="char" arraysize="*" 
        value="...@..." utype="voprov:Agent.email"/>
    <GROUP name="InputParams" utype="voprov:Parameter">
      <PARAM ID="RGB" arraysize="*" datatype="char" name="RGB" 
          type="no_query" value="RGB.jpg">
        <DESCRIPTION>RGB image name</DESCRIPTION>
      </PARAM>
      <PARAM ID="order" arraysize="*" datatype="char" name="order" 
          type="no_query" value="RGB">
        <DESCRIPTION>order of the channels</DESCRIPTION>
        <VALUES>
          <OPTION value="RGB"/>
          <OPTION value="RBG"/>
          <OPTION value="GBR"/>
          <OPTION value="GRB"/>
          <OPTION value="BRG"/>
          <OPTION value="BGR"/>
        </VALUES>
      </PARAM>
    </GROUP>
    <GROUP name="Used" utype="voprov:Used">
      <PARAM arraysize="*" datatype="char" name="R" 
          value="R.jpg" utype="voprov:Entity.id" xtype="image/jpeg">
        <DESCRIPTION>Image for red channel</DESCRIPTION>
      </PARAM>
      <PARAM arraysize="*" datatype="char" name="G" 
          value="G.jpg" utype="voprov:Entity.id" xtype="image/jpeg">
        <DESCRIPTION>Image for green channel</DESCRIPTION>
      </PARAM>
      <PARAM arraysize="*" datatype="char" name="B"
          value="B.jpg" utype="voprov:Entity.id" xtype="image/jpeg">
        <DESCRIPTION>Image for blue channel</DESCRIPTION>
      </PARAM>
    </GROUP>
    <GROUP name="Generated" utype="voprov:WasGeneratedBy">
      <PARAM arraysize="*" datatype="char" name="RGB" ref="RGB"
          value="RGB.jpg" utype="voprov:Entity.id"  xtype="image/jpeg">
        <DESCRIPTION>RGB image name</DESCRIPTION>
      </PARAM>
    </GROUP>
  </RESOURCE>
</VOTABLE>

\end{verbnobox}

\subsection{W3C PROV-DM compatible serializations}\label{sec:w3cserialization}

According to our minimum requirements (see Section~\ref{sec:requirements}), it must be possible to
serialize the provenance metadata into a format compatatible with the W3C Provenance Data Model (W3C PROV-DM), so that it can be exchanged within a wider context and can be processed by already existing tools, e.g. for visualizing provenance.
W3C PROV-DM is a larger set of classes and relations compared to this model, but sharing the same core structure. It allows the possibility to add IVOA or \textit{ad hoc} attributes to the basic ones in each class. Thus we can add our additional attributes without problems and still be W3C conform. However, we also defined a few additional classes and relations that are not W3C conform and thus need to be restructured.
% In our data model we have defined additional classes and attributes that are not W3C conform and thus need to be serialized with different names/structures.
Using ``voprov'' as the namespace prefix for our model and ``prov'' for W3C PROV-DM, the necessary changes for mapping from ProvenanceDM to W3C PROV-DM are listed in the following paragraphs.

\paragraph{Mapping of classes and attributes}
\begin{itemize}
\item namespace \texttt{voprov} $\rightarrow$ \texttt{prov} for those attributes that are the same in W3C (e.g. ID, role, startTime, endTime)
\item attribute \texttt{voprov:name} $\rightarrow$ \texttt{prov:label}
\item attribute \texttt{voprov:annotation} $\rightarrow$ \texttt{prov:description}
\item attribute \texttt{prov:role} is not allowed in W3C's \class{WasAttributedTo}, thus use \texttt{voprov:role}
\item \class{hadMember} has no ID and no optional attributes in W3C
\item \class{Collection} $\rightarrow$ \class{Entity} with \texttt{prov:type = prov:collection}
\end{itemize}

\paragraph{Description classes}
%\item restructure \class{*Description} classes: add their attributes to the linked core classes, using a ``desc\_'' prefix, e.g. Entity.desc\_category.
\begin{itemize}
\item \class{ActivityDescription} becomes an Entity with prov:type = voprov:ActivityDescription. It is close to the Plan concept in W3C PROV, and it could have in addition prov:type = prov:plan (note that W3C PROV accepts several types). This class is used by the Activity class with prov:role = voprov:ActivityDescription.
\item EntityDescription becomes an Entity of type prov:type = voprov:EntityDescription. It is then linked to an Entity with the sepcializationOf relation.
\item UsedDescription and WasGeneratedBy description also become entities.
\end{itemize}
We also envision to group all description classes into a prov:Bundle that is then connected to the Activity with the Used relation and prov:role = voprov:ActivityDescription.

\paragraph{Parameter class}
%\item restructure \class{Parameter} and \class{ParameterDescription}:
%merge them into one parameter class, model it as an entity
A parameter can be seen as a simplified entity and should thus be serialized in the same way as an entity, with prov:type = voprov:Parameter. The ParameterDescription becomes an Entity with prov:type = voprov:ParameterDescription and a specializationOf relation to the parameter entity.

\paragraph{ActivityFlow class}
\begin{itemize}
	\item \class{ActivityFlow} $\rightarrow$ \class{Activity} with additional attribute \texttt{voprov:votype = 'voprov:activityFlow'}
	\item replace \class{HadStep} relation by W3C's general \class{WasInfluencedBy} relation with additional attribute \texttt{voprov:votype = 'voprov:hadStep'} or just use \attribute{voprov:hadStep} as attribute in activities of type activityFlow
\end{itemize}

This way, it is possible to produce W3C compatible serializations of our model with minimum information loss. W3C tools would ignore the voprov-attributes, whereas VO clients could make sense of this additional information and could even uncover the original structure or convert it to a VO serialization.
