\subsection{W3C PROV-DM compatible serializations}\label{sec:w3cserialization}
According to our minimum requirements (see Section~\ref{sec:requirements}), it must be possible to
serialize the provenance metadata into a format compatatible with the W3C Provenance Data Model (W3C PROV-DM), so that it can be exchanged within a wider context and can be processed by already existing tools, e.g. for visualizing provenance.
W3C PROV-DM is a larger set of classes and relations compared to this model, but sharing the same core structure. It allows the possibility to add IVOA or \textit{ad hoc} attributes to the basic ones in each class. Thus we can add our additional attributes without problems and still be W3C conform. However, we also definved a few additional classes and relations that are not W3C conform and thus need to be restructured.
% In our data model we have defined additional classes and attributes that are not W3C conform and thus need to be serialized with different names/structures.
Using ``voprov'' as the namespace prefix for our model and ``prov'' for W3C PROV-DM, the necessary changes for mapping from ProvenanceDM to W3C PROV-DM are:

\begin{itemize}
\item namespace \texttt{voprov} $\rightarrow$ \texttt{prov} for those attributes that are the same in W3C (e.g. ID, role, startTime, endTime)
\item attribute \texttt{voprov:name} $\rightarrow$ \texttt{prov:label}
\item attribute \texttt{voprov:annotation} $\rightarrow$ \texttt{prov:description}
\item attribute \texttt{prov:role} is not allowed in W3C's \emph{wasAttributedTo}, thus use \texttt{voprov:role}
\item \emph{hadMember} has no ID and no optional attributes in W3C
\item \emph{Collection} $\rightarrow$ \emph{Entity} with \texttt{prov:type = prov:collection}
\item restructure \emph{ActivityFlow}:
	\begin{itemize}
	\item \emph{ActivityFlow} $\rightarrow$ \emph{Activity} with additional attribute \texttt{voprov:votype = 'voprov:activityFlow'}
	\item replace \emph{hadStep} relation by W3C's general \emph{wasInfluencedBy} relation with additional attribute \texttt{voprov:votype = 'voprov:hadStep'}
	\end{itemize}
\item restructure \emph{*Description} classes: add their attributes to the linked core classes, using a ``desc\_'' prefix, e.g. Entity.desc\_category.

\end{itemize}

This way, it is possible to produce W3C compatible serializations of our model with minimum information loss. W3C tools would ignore the voprov-attributes, whereas VO clients could make sense of this additional information and could even uncover the original structure or convert it to a VO serialization.




