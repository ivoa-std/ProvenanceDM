\subsection{W3C PROV-DM compatible serializations}\label{sec:w3cserialization}
According to our minimum requirements (see Section~\ref{sec:requirements}), it must be possible to
serialize the provenance metadata into a format compatatible with the W3C Provenance Data Model (W3C PROV-DM), so that it can be exchanged within a wider context and can be processed by already existing tools, e.g. for visualizing provenance.
W3C PROV-DM is a larger set of classes and relations compared to this model, but sharing the same core structure. It allows the possibility to add IVOA or \textit{ad hoc} attributes to the basic ones in each class. Thus we can add our additional attributes without problems and still be W3C conform. However, we also definved a few additional classes and relations that are not W3C conform and thus need to be restructured.
% In our data model we have defined additional classes and attributes that are not W3C conform and thus need to be serialized with different names/structures.
Using ``voprov'' as the namespace prefix for our model and ``prov'' for W3C PROV-DM, the necessary changes for mapping from ProvenanceDM to W3C PROV-DM are listed in the following paragraphs

\paragraph{Mapping of classes and attributes}
\begin{itemize}
\item namespace \texttt{voprov} $\rightarrow$ \texttt{prov} for those attributes that are the same in W3C (e.g. ID, role, startTime, endTime)
\item attribute \texttt{voprov:name} $\rightarrow$ \texttt{prov:label}
\item attribute \texttt{voprov:annotation} $\rightarrow$ \texttt{prov:description}
\item attribute \texttt{prov:role} is not allowed in W3C's \class{WasAttributedTo}, thus use \texttt{voprov:role}
\item \class{hadMember} has no ID and no optional attributes in W3C
\item \class{Collection} $\rightarrow$ \class{Entity} with \texttt{prov:type = prov:collection}
\end{itemize}

\paragraph{Description classes}
%\item restructure \class{*Description} classes: add their attributes to the linked core classes, using a ``desc\_'' prefix, e.g. Entity.desc\_category.
\begin{itemize}
\item \class{ActivityDescription} becomes an Entity with prov:type = voprov:ActivityDescription. It is close to the Plan concept in W3C PROV, and it could have in addition prov:type = prov:plan (note that W3C PROV accepts several types). This class is used by the Activity class with prov:role = voprov:ActivityDescription.
\item EntityDescription becomes an Entity of type prov:type = voprov:EntityDescription. It is then linked to an Entity with the sepcializationOf relation.
\item UsedDescription and WasGeneratedBy description also become entities.
\end{itemize}
We also envision to group all description classes into a prov:Bundle that is then connected to the Activity with the Used relation and prov:role = voprov:ActivityDescription.

\paragraph{Parameter class}
%\item restructure \class{Parameter} and \class{ParameterDescription}:
%merge them into one parameter class, model it as an entity
A parameter can be seen as a simplified entity and should thus be serialized in the same way as an entity, with prov:type = voprov:Parameter. The ParameterDescription becomes an Entity with prov:type = voprov:ParameterDescription and a specializationOf relation to the parameter entity.

\paragraph{ActivityFlow class}
\begin{itemize}
	\item \class{ActivityFlow} $\rightarrow$ \class{Activity} with additional attribute \texttt{voprov:votype = 'voprov:activityFlow'}
	\item replace \class{HadStep} relation by W3C's general \class{WasInfluencedBy} relation with additional attribute \texttt{voprov:votype = 'voprov:hadStep'} or just use \attribute{voprov:hadStep} as attribute in activities of type activityFlow
\end{itemize}

This way, it is possible to produce W3C compatible serializations of our model with minimum information loss. W3C tools would ignore the voprov-attributes, whereas VO clients could make sense of this additional information and could even uncover the original structure or convert it to a VO serialization.




