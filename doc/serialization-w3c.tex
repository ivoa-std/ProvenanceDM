\subsection{W3C PROV-DM compatible serializations}
According to our minimum requirements (see Section~\ref{sec:requirements}), it must be possible to
serialize the provenance metadata into the W3C compatible formats, so that it can be exchanged within a wider context and can processed by already existing tools.
In our data model we have defined additional classes and attributes that are not W3C conform and thus need to be
serialized with different names/structures. Using ``voprov'' as the namespace prefix for our model and ``prov'' for W3C PROV-DM, the necessary changes for mapping from ProvenanceDM to W3C PROV-DM are:

\begin{itemize}
\item namespace \texttt{voprov} $\rightarrow$ \texttt{prov} for those attributes that are the same in W3C (e.g. ID, role, startTime, endTime)
\item attribute \texttt{voprov:name} $\rightarrow$ \texttt{prov:label}
\item attribute \texttt{voprov:annotation} $\rightarrow$ \texttt{prov:description}
\item attribute \texttt{prov:role} is not allowed in W3C's \emph{wasAttributedTo}, thus use \texttt{voprov:role}
\item \emph{hadMember} has no ID and no optional attributes in W3C
\item \emph{Collection} $\rightarrow$ \emph{Entity} with \texttt{prov:type = prov:collection}
\item restructure \emph{ActivityFlow}:
	\begin{itemize}
	\item \emph{ActivityFlow} $\rightarrow$ \emph{Activity} with additional attribute \texttt{voprov:votype = 'voprov:activityFlow'}
	\item replace \emph{hadStep} relation by W3C's general \emph{wasInfluencedBy} relation with additional attribute \texttt{voprov:votype = 'voprov:hadStep'}
	\end{itemize}

\end{itemize}

This way, one can produce W3C compatible serializations of our model with minimum information loss.
